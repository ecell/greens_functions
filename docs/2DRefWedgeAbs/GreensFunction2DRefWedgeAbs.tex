\documentclass{article}
\usepackage{amsmath}
\usepackage{mathrsfs}

\begin{document}

\title{GreensFunction2DRefWedgeAbs}
\author{Toru Niina}
\maketitle

\section{Overview}

System contains only one particle. 
 Position is described by cylindrical coordinate system.
 Initial position of the particle is $r = r_0$, $\theta = \theta_0$.

And boundary conditions are
\begin{eqnarray}
    P(a, \theta, t) &=& 0 \nonumber \\
    \left.\frac{\partial P}{\partial\theta}\right|_{\theta = 0} &=& 0 \nonumber \\
    \left.\frac{\partial P}{\partial\theta}\right|_{\theta = \phi} &=& 0
    \label{boundary-conditions}
\end{eqnarray}
where $0 < \phi < 2\pi$.

When the condition $\theta_0 = \frac{\phi}{2}$ is satisfied, 
these setups become equivalent to the condition that
the diffusion on the lateral surface of infinite-height cone.
It should be noted that the code GreensFunction2DRefWedgeAbs
is optimized for this case($\theta_0 = \frac{\phi}{2}$), so
you cannot set $\theta_0$ in the code.
See section \ref{sec:applying-surface-of-cone}.

\subsection{Green's Function}
General form
\begin{eqnarray}
    & & G(r, \theta, t | r_0, \theta_0, 0) \nonumber \\
    &=& \frac{2}{\phi a^2} \sum^{\infty}_{m=1}
        \mathrm{e}^{-\alpha_{m0}^2Dt}
        \frac{J_0(r_0\alpha_{m0})
              J_0(r\alpha_{m0})}
             {\{J_1(a\alpha_{m0})\}^2}+\nonumber\\
    & & \frac{4}{\phi a^2}\sum^{\infty}_{n=1}
        \cos\frac{n\pi\theta}{\phi}\cos\frac{n\pi\theta_0}{\phi}
        \sum^{\infty}_{m=1}
        \mathrm{e}^{-\alpha_{mn}^2Dt}
        \frac{J_{\frac{n\pi}{\phi}}(r_0\alpha_{mn})
              J_{\frac{n\pi}{\phi}}(r\alpha_{mn})}
             {\{J'_{\frac{n\pi}{\phi}}(a\alpha_{mn})\}^2}
    \label{greens-function}
\end{eqnarray}

Optimized form
\begin{eqnarray}
    & & G_{opt}(r, \theta, t | r_0, \frac{\phi}{2}, 0) \nonumber \\
    &=& \frac{2}{\phi a^2} \sum^{\infty}_{m=1}
        \mathrm{e}^{-\alpha_{m0}^2Dt}
        \frac{J_0(r_0\alpha_{m0})
              J_0(r\alpha_{m0})}
             {\{J_1(a\alpha_{m0})\}^2}+\nonumber\\
    & & \frac{4}{\phi a^2}\sum^{\infty}_{n=1}
        (-1)^n\cos\frac{2n\pi\theta}{\phi}
        \sum^{\infty}_{m=1}
        \mathrm{e}^{-\alpha_{mn}^2Dt}
        \frac{J_{\frac{2n\pi}{\phi}}(r_0\alpha_{mn})
              J_{\frac{2n\pi}{\phi}}(r\alpha_{mn})}
             {\{J'_{\frac{2n\pi}{\phi}}(a\alpha_{mn})\}^2}
\end{eqnarray}

\subsection{Derivation of the Green's function}
As a starting point, consider 2 dimentional diffusion equation in infinite space
and the solution of that.

\begin{eqnarray}
    \frac{\partial u}{\partial t} &=& D\nabla^2 u \\
    u(R, t) &=& \frac{1}{4\pi Dt}\mathrm{exp}\left(\frac{-R^2}{4Dt}\right)
\end{eqnarray}
here, $R$ is a distance from initial position $(x_0, y_0)$ to $(x, y)$, so
\begin{equation}
 R = \sqrt{(x-x_0)^2+(y-y_0)^2}.
\end{equation}

When the equation is considered in Cylindrical coordinate, the form of solution
 is same as above but the definition of distance $R$ become

\begin{equation}
    R = \sqrt{r^2+r_0^2+rr_0\cos(\theta - \theta_0)}.
\end{equation}

Let the solution of diffusion equation $v$ under the boundary conditions
 (\ref{boundary-conditions}) as
\begin{equation}
    v = u + w_\theta + w_r.
\end{equation}
here, $u$ corresponds to the solution of diffusion equation in the case of
 infinite space. Each $w_i$ satisfies the same equation and cancels the value
 of $u$ on the boundary.

To solve this, use Laplace transformation.
 The solution of diffusion equation in infinite space with transformation
 $(t\rightarrow p)$ is

\begin{equation} \label{Laplace-u-solution-infinite}
    \bar{u} = \frac{1}{2\pi D}K_0(qR)
\end{equation}
here, q is defined as $q = \sqrt{\cfrac{p}{D}}$ and $K_0$ means modified Bessel
 function of the second kind.

Now, it can be written that
\begin{equation}\label{expansion-K0}
    K_0(qR) = P.V.\int^{\infty i}_{-\infty i}
              \frac{\cos\nu (\pi - \theta + \theta_0)}{\sin\nu\pi}
              K_\nu(qr)I_\nu(qr_0)id\nu
\end{equation}
where $0 < \theta - \theta_0 < 2\pi$ and $r_0 < r$.
 Here, $P.V.$ means Cauthy's principal value of the origin.

Then $\bar{u}$ can be re-written as
\begin{equation}\label{expansion-bar-u}
    \bar{u} = \frac{i}{2\pi D}P.V.\int^{\infty i}_{-\infty i}
              \frac{\cos\nu (\pi - \theta + \theta_0)}{\sin\nu\pi}
              K_\nu(qr)I_\nu(qr_0)d\nu
\end{equation}
where $0 < \theta - \theta_0 < 2\pi$ and $r_0 < r$.

At first, determine the form of $w_\theta$. 
In order to cancel out the value of $\cfrac{\partial\bar{u}}{\partial\theta}$
on the boundary, where $\theta = \phi$ and $r_0 < r$, $\bar{w_\theta}$
should satisfy below.
\begin{eqnarray}
    \left.\frac{\partial\bar{w_\theta}}{\partial\theta}\right|_{\theta = \phi}
    &=& \frac{-i}{2\pi D}P.V.\int^{\infty i}_{-\infty i}
              \frac{\nu\sin\nu(\pi - \phi + \theta_0)}{\sin\nu\pi}
              K_\nu(qr)I_\nu(qr_0)d\nu\label{boundary-value-phi}\\
    \left.\frac{\partial\bar{w_\theta}}{\partial\theta}\right|_{\theta = 0}
    &=& \frac{-i}{2\pi D}P.V.\int^{\infty i}_{-\infty i}
              \frac{-\nu\sin\nu(\pi - \theta_0)}{\sin\nu\pi}
              K_\nu(qr)I_\nu(qr_0)d\nu \label{boundary-value-zero}
\end{eqnarray}

With the Laplace transformation $t\rightarrow p$, subsidiary equation become
\begin{eqnarray}
    D\left( \frac{\partial^2}{\partial r^2} +
            \frac{1}{r}\frac{\partial}{\partial r} +
            \frac{1}{r^2}\frac{\partial^2}{\partial \theta^2}
            \right)
    \bar{\Psi}
    &=&
    p\bar{\Psi} \nonumber \\
    r^2 \frac{\partial^2}{\partial r^2}\bar{\Psi} +
    r   \frac{\partial}{\partial r}\bar{\Psi} +
        \frac{\partial^2}{\partial \theta^2}\bar{\Psi} -
    r^2 \frac{p}{D}\bar{\Psi} &=& 0.
\end{eqnarray}

Assuming that the form of the solution $\Psi$ can be written as
 $R(r, t)\Theta(\theta)$ and $\Theta(\theta)$ satisfies the equation below, 
\begin{equation}
    \frac{d^2}{d\theta^2}\bar{\Theta} = -n^2\bar{\Theta}
\end{equation}

The subsidialy equation become the same form as Modified Bessel's equation.
\begin{eqnarray} \label{subsidiary-eq}
    r^2\frac{\partial^2}{\partial r^2}\bar{\Psi} +
    r\frac{\partial}{\partial r}\bar{\Psi} -(n^2 + r^2q^2)\bar{\Psi} &=& 0.
\end{eqnarray}
So the solution of (\ref{subsidiary-eq}) is a multipletation between
 a linear combination of Modified Bessel function, $I_n(qr), K_n(qr)$ and
 a linear combination of $\sin\theta$ and $\cos\theta$.

\begin{equation}\label{solution-of-subsidialy-equation}
    solution = \sum_n\Bigl( A\cos n\theta + B\sin n\theta\Bigr)
    \Bigl(CI_n(rq) + DK_n(rq)\Bigr)
\end{equation}

Here, this integral
\begin{equation}\label{general-integral-form-of-w}
P.V.\int^{\infty i}_{-\infty i}
    \frac{\cos\nu\pi}{\sin\nu\pi}
    (A\cos\nu(\phi - \theta) + B\cos\nu\theta)
    K_\nu(qr)id\nu
\end{equation}
% is evaluated as
% \begin{equation}
%     \sum_n\Bigl( (A\cos n\phi + B)\cos n\theta +
%                   A\sin n\phi\sin n\theta\Bigr)K_n(qr)
% \end{equation}
% So the form (\ref{general-integral-form-of-w})
satisfies (\ref{subsidiary-eq}).

Assuming that $\bar{w_\theta}$ is
\begin{equation}
    \bar{w_\theta}
  = \frac{-i}{2\pi D}P.V.\int^{\infty i}_{-\infty i}
    \frac{\cos\nu\pi}{\sin\nu\pi}
    (A\cos\nu(\phi - \theta) + B\cos\nu\theta)
    K_\nu(qr)d\nu
\end{equation}
and determine the constants $A$ and $B$. From boundary condition
(\ref{boundary-value-phi}) and (\ref{boundary-value-zero})
\begin{eqnarray}
%
% theta = phi case.
%
    \left.\frac{\partial\bar{w_\theta}}{\partial\theta}\right|_{\theta = \phi}
    &=& \frac{i}{2\pi D}P.V.\int^{\infty i}_{-\infty i}
    \frac{\cos\nu\pi}{\sin\nu\pi}
    \nu B\sin\nu\phi
    K_\nu(qr)d\nu \nonumber \\
    &=& \frac{-i}{2\pi D}P.V.\int^{\infty i}_{-\infty i}
        \frac{\nu\sin\nu(\pi - \phi + \theta_0)}{\sin\nu\pi}
        K_\nu(qr)I_\nu(qr_0)d\nu \nonumber \\
%
% theta = zero case.
%
    \left.\frac{\partial\bar{w_\theta}}{\partial\theta}\right|_{\theta = 0}
    &=& \frac{-i}{2\pi D}P.V.\int^{\infty i}_{-\infty i}
    \frac{\cos\nu\pi}{\sin\nu\pi}
    \nu A\sin\nu\phi
    K_\nu(qr)d\nu \nonumber \\
    &=& \frac{-i}{2\pi D}P.V.\int^{\infty i}_{-\infty i}
        \frac{-\nu\sin\nu(\pi - \theta_0)}{\sin\nu\pi}
        K_\nu(qr)I_\nu(qr_0)d\nu
\end{eqnarray}
thus $A$ and $B$ are evaluated as
\begin{eqnarray}
    A &=& \frac{-\sin\nu(\pi - \theta_0)}{\cos\nu\pi\sin\nu\phi}I_\nu(qr_0)
    \nonumber \\
    B &=& \frac{-\sin\nu(\pi-\phi+\theta_0)}{\cos\nu\pi\sin\nu\phi}I_\nu(qr_0)
    \nonumber \\
\end{eqnarray}
So $\bar{w_\theta}$ is
\begin{eqnarray}
    \bar{w_\theta} &=& \frac{i}{2\pi D}P.V.\int^{\infty i}_{-\infty i}
                \frac{\sin\nu(\pi - \theta_0)\cos\nu(\phi - \theta) +
                      \sin\nu(\pi-\phi+\theta_0)\cos\nu\theta}
                {\sin\nu\pi\sin\nu\phi}\nonumber \\
            & & K_\nu(qr)I_\nu(qr_0)d\nu\label{bar-w-theta}
\end{eqnarray}

And now, $\bar{w_r}$ should satisfy the same equation and the same condition for 
$\theta$, the form of $\bar{w_r}$ can be obtained as 
\begin{eqnarray}
    \bar{w_r} &=& \frac{-i}{\pi D}P.V.\int^{\infty i}_{-\infty i}
                  \frac{\cos\nu\theta_0\cos\nu(\phi - \theta)}{\sin\nu\phi}
                  \frac{I_\nu(qr)K_\nu(qa)I_\nu(qr_0)}{I_\nu(qa)}d\nu
    \label{bar-w-r}
\end{eqnarray}

Then $\bar{v}$ can be evaluated by adding 
(\ref{expansion-bar-u}), (\ref{bar-w-theta}) and (\ref{bar-w-r}).

\begin{eqnarray} 
    \bar{v} &=& \bar{u} + \bar{w_\theta} + \bar{w_r} \nonumber \\
            &=& \frac{i}{\pi D}P.V.\int^{\infty i}_{-\infty i}
                \frac{\cos\nu\theta_0\cos\nu(\phi - \theta)}{\sin\nu\phi}
                \times\nonumber\\
            & & \frac{I_\nu(qr_0)\left(K_\nu(qr)I_\nu(qa) - K_\nu(qa)I_\nu(qr)
                                 \right)}
                     {I_\nu(qa)}
                d\nu
    \label{bar-v}
\end{eqnarray}

Now, let
\begin{equation}
    s = n\pi / \phi
\end{equation}
and define
\begin{equation}
    F_\nu(x, y) = I_\nu(x)K_\nu(y) - K_\nu(x)I_\nu(y)
\end{equation}

then the values of residues are
\begin{eqnarray}
    & & \frac{-i}{\pi D}\lim_{\nu\to s}\left[
        (\nu - s)\frac{\cos\nu\theta_0\cos\nu(\phi - \theta)}{\sin\nu\phi}
                 \frac{I_\nu(qr_0)F_\nu(qr, qa)}{I_\nu(qa)}
        \right] \nonumber \\
    &=& \frac{-i}{\pi D}\lim_{\nu\to s}\left[
        \frac{\nu\phi - s\phi}{\phi\sin(\nu\phi-s\phi)}
        \cos\nu\theta_0\cos\nu(\phi - \theta)
        \frac{I_\nu(qr_0)F_\nu(qa, qr)}{I_\nu(qa)}
        \right] \nonumber \\
    &=& \frac{1}{\pi iD\phi}
        \cos\frac{n\pi\theta_0}{\phi}\cos\frac{n\pi\theta}{\phi}
        \frac{I_{\frac{n\pi}{\phi}}(qr_0)F_{\frac{n\pi}{\phi}}(qa, qr)}
        {I_{\frac{n\pi}{\phi}}(qa)}\nonumber
\end{eqnarray}

and at $\nu = 0$, hoge hoge

so based on Cauchy's theorem
\begin{eqnarray}
    \bar{v} = \frac{1}{D\phi}
      &\Biggl[& \frac{I_0(qr_0)F_0(qr, qa)}{I_0(qa)} + \nonumber\\
      &       & 2\sum^{\infty}_{n=1}
                \cos\frac{n\pi\theta}{\phi}\cos\frac{n\pi\theta_0}{\phi}
                \frac{I_{\frac{n\pi}{\phi}}(qr_0)F_{\frac{n\pi}{\phi}}(qa, qr)}
                {I_{\frac{n\pi}{\phi}}(qa)}\Biggr]
\end{eqnarray}

To evaluate $v$, consider Bromwich integral of $\bar{v}$, 
\begin{eqnarray}
    v &=& \frac{1}{D\phi}\frac{1}{2\pi i}\int
    \Biggl[ \frac{I_0(qr_0)F_0(qr, qa)}{I_0(qa)} + \nonumber\\
      & &  2\sum^{\infty}_{n=1}
           \cos\frac{n\pi\theta}{\phi}\cos\frac{n\pi\theta_0}{\phi}
           \frac{I_{\frac{n\pi}{\phi}}(qr_0)F_{\frac{n\pi}{\phi}}(qa, qr)}
           {I_{\frac{n\pi}{\phi}}(qa)}\exp(pt)dp\Biggr] \nonumber \\
      &=&  \frac{1}{D\phi}\frac{1}{2\pi i}\int
           \frac{I_0(qr_0)F_0(qr, qa)}{I_0(qa)}\exp(pt)dp +\nonumber\\
      & &  \frac{2}{D\phi}\sum^{\infty}_{n=1}
           \cos\frac{n\pi\theta}{\phi}\cos\frac{n\pi\theta_0}{\phi}
           \frac{1}{2\pi i}\int
           \frac{I_{\frac{n\pi}{\phi}}(qr_0)F_{\frac{n\pi}{\phi}}(qa, qr)}
           {I_{\frac{n\pi}{\phi}}(qa)}\exp(pt)dp
    \label{v-bromwich-integral-form}
\end{eqnarray}

Using variable transformation $q = -is \to p = -Ds^2$, the form of the function,
the first term integral
\begin{eqnarray}
      \int\frac{I_0(qr_0)F_0(qr, qa)}{I_0(qa)}\exp(pt)dp
\end{eqnarray}
 become
\begin{eqnarray}
      \int\frac{I_0(-isr_0)}{I_0(-isa)}
      (I_\nu(-isr)K_\nu(-isa) - K_\nu(-isr)I_\nu(-isa))\exp(-s^2Dt)(-2Ds)ds
\end{eqnarray}
And re-writing the above integral by using characteristics of modified Bessel function
\begin{eqnarray}
    I_n(-ix) &=& i^{-n}J_n(x)\nonumber \\
    K_n(-ix) &=& \frac{\pi}{2}i^{n+1}(J_n(x)+iY_n(x))\nonumber
\end{eqnarray}
with some rearrangement, the integral become
\begin{eqnarray}
    & & \int\frac{I_0(-isr_0)}{I_0(-isa)}
        (I_\nu(-isr)K_\nu(-isa) - K_\nu(-isr)I_\nu(-isa))\exp(-s^2Dt)(-2Ds)ds
        \nonumber \\
    &=& -2D\int s\frac{J_0(sr_0)}{J_0(sa)}
        (i^{-\nu}J_\nu(sr)K_\nu(-isa) - K_\nu(-isr)i^{-\nu}J_\nu(sa))\exp(-s^2Dt)ds
        \nonumber \\
    &=& -2D\int s\frac{J_0(sr_0)}{J_0(sa)}
        (i^{-\nu}J_\nu(sr)\frac{\pi}{2}i^{\nu+1}(J_\nu(sa)+iY_\nu(sa)) - \nonumber \\
    & & \frac{\pi}{2}i^{\nu+1}(J_\nu(sr)+iY_\nu(sr))i^{-\nu}J_\nu(sa))\exp(-s^2Dt)ds
        \nonumber \\
    &=& -2D\frac{\pi}{2}\int s\frac{J_0(sr_0)}{J_0(sa)}
        (J_\nu(sr)(iJ_\nu(sa)-Y_\nu(sa)) - \nonumber \\
    & & (iJ_\nu(sr)-Y_\nu(sr))J_\nu(sa))\exp(-s^2Dt)ds
        \nonumber \\
    &=& \int sD\pi\frac{J_0(sr_0)}{J_0(sa)}
        (J_\nu(sr)Y_\nu(sa) - Y_\nu(sr)J_\nu(sa))\exp(-s^2Dt)ds
        \nonumber \\
    & & \int sD\pi\frac{J_0(sr_0)}{J_0(sa)}
        (J_0(sr)Y_0(sa) - Y_0(sr)J_0(sa))\exp(-s^2Dt)ds
        \nonumber
\end{eqnarray}
and in the same way, second term become
\begin{eqnarray}
    & & \int\frac{I_{\frac{n\pi}{\phi}}(qr_0)F_{\frac{n\pi}{\phi}}(qa, qr)}
        {I_{\frac{n\pi}{\phi}}(qa)}\exp(pt)dp \nonumber \\
    &=& \int sD\pi\frac{J_{\frac{n\pi}{\phi}}(sr_0)}{J_{\frac{n\pi}{\phi}}(sa)}
        (J_{\frac{n\pi}{\phi}}(sr)Y_{\frac{n\pi}{\phi}}(sa) -
         Y_{\frac{n\pi}{\phi}}(sr)J_{\frac{n\pi}{\phi}}(sa))\exp(-s^2Dt)ds
        \nonumber
\end{eqnarray}
the poles of these functions are at the point that $s$ satisfies
$J_{\frac{n\pi}{\phi}}(sa)$, $0 \leq n$.

Now let $a\alpha_{mn}$ is m-th root of $J_{\frac{n\pi}{\phi}}$,
then the value of each residues become
\begin{equation}
    \lim_{s\to\alpha_{mn}} \left[
       (s-\alpha_{mn})
       \pi Ds\mathrm{e}^{-s^2Dt}
       \frac{J_{\frac{n\pi}{\phi}}(sr_0)}{J_{\frac{n\pi}{\phi}}(sa)}
       \left(J_{\frac{n\pi}{\phi}}(sa)Y_{\frac{n\pi}{\phi}}(sr) -
        J_{\frac{n\pi}{\phi}}(sr)Y_{\frac{n\pi}{\phi}}(sa)\right)
       \right]
\end{equation}

In the same way as GreensFunction2DAbs, this can be evaluated as below.
From the definition of differenciation,
\begin{eqnarray}
    \lim_{s\to\alpha_{mn}}\frac{(s-\alpha_{mn})}{J_{\frac{n\pi}{\phi}}(as)}
    &=& \frac{1}{a} \lim_{s\to\alpha_{mn}} \frac{(as-a\alpha_{mn})}
        {J_{\frac{n\pi}{\phi}}(as) - J_{\frac{n\pi}{\phi}}(a\alpha_{mn})}\nonumber \\
    &=& \frac{1}{aJ'_{\frac{n\pi}{\phi}}(a\alpha_{mn})}\nonumber\\
    &=& \frac{2}{a(J_{\frac{n\pi}{\phi}-1}(a\alpha_{mn}) -
                   J_{\frac{n\pi}{\phi}+1}(a\alpha_{mn}))}\nonumber
\end{eqnarray}

And by definition of $\alpha_{mn}$
\begin{eqnarray}
    & & \lim_{s\to\alpha_{mn}}-J_{\frac{n\pi}{\phi}}(sr_0)
        (J_{\frac{n\pi}{\phi}}(sa)Y_{\frac{n\pi}{\phi}}(sr) -
         J_{\frac{n\pi}{\phi}}(sr)Y_{\frac{n\pi}{\phi}}(sa)) \nonumber \\
    &=& -J_{\frac{n\pi}{\phi}}(r_0\alpha_{mn})
         J_{\frac{n\pi}{\phi}}(r\alpha_{mn})Y_{\frac{n\pi}{\phi}}(a\alpha_{mn})
\end{eqnarray}

So the values of residues are
\begin{equation}\label{residue-values}
    Res_m = 
    \pi D\alpha_{mn}\mathrm{e}^{-\alpha_{mn}^2Dt}
    \frac{J_{\frac{n\pi}{\phi}}(r_0\alpha_{mn})}
         {aJ'_{\frac{n\pi}{\phi}}(a\alpha_{mn})}
       \left(-J_{\frac{n\pi}{\phi}}(r\alpha_{mn})
         Y_{\frac{n\pi}{\phi}}(a\alpha_{mn})\right)
\end{equation}

Additionally, the characteristics of Bessel functions
\begin{eqnarray}
    J_n(x)Y'_n(x) - J'_n(x)Y_n(x) = \frac{2}{\pi x} \nonumber \\
    Y_n(x) = \frac{-2}{\pi xJ'_n(x)} + \frac{J_n(x)Y'_n(x)}{J'_n(x)}
\end{eqnarray}
(\ref{residue-values}) is re-writtened as
\begin{eqnarray}
% procedure of derivation
   Res_m &=& \pi D\alpha_{mn}\mathrm{e}^{-\alpha_{mn}^2Dt}
   \frac{J_{\frac{n\pi}{\phi}}(r_0\alpha_{mn})}
        {aJ'_{\frac{n\pi}{\phi}}(a\alpha_{mn})}
      \left(-J_{\frac{n\pi}{\phi}}(r\alpha_{mn})
      \frac{-2}{\pi a\alpha_{mn}J'_{\frac{n\pi}{\phi}}(a\alpha_{mn})}\right) \nonumber \\
   &=& D\mathrm{e}^{-\alpha_{mn}^2Dt}
   \frac{J_{\frac{n\pi}{\phi}}(r_0\alpha_{mn})}
        {aJ'_{\frac{n\pi}{\phi}}(a\alpha_{mn})}
      J_{\frac{n\pi}{\phi}}(r\alpha_{mn})
      \frac{2}{aJ'_{\frac{n\pi}{\phi}}(a\alpha_{mn})} \nonumber \\
    Res_m &=& 2D\mathrm{e}^{-\alpha_{mn}^2Dt}
    \frac{J_{\frac{n\pi}{\phi}}(r_0\alpha_{mn})
          J_{\frac{n\pi}{\phi}}(r\alpha_{mn})}
         {a^2\{J'_{\frac{n\pi}{\phi}}(a\alpha_{mn})\}^2}
    \label{residue-values-only-J}
\end{eqnarray}

Then the integral is evaluated as
\begin{eqnarray}
    2\pi i\sum Res_m &=& 2\pi i\sum2D\mathrm{e}^{-\alpha_{mn}^2Dt}
    \frac{J_{\frac{n\pi}{\phi}}(r_0\alpha_{mn})
          J_{\frac{n\pi}{\phi}}(r\alpha_{mn})}
         {a^2\{J'_{\frac{n\pi}{\phi}}(a\alpha_{mn})\}^2}
\end{eqnarray}
with substituting this into (\ref{v-bromwich-integral-form}), $v$ is 
\begin{eqnarray}
    v &=& \frac{1}{D\phi}
          \sum^{\infty}_{m=1}
          2D\mathrm{e}^{-\alpha_{m0}^2Dt}
          \frac{J_0(r_0\alpha_{m0})
                J_0(r\alpha_{m0})}
               {a^2\{J'_0(a\alpha_{m0})\}^2}+\nonumber\\
      & & \frac{2}{D\phi}\sum^{\infty}_{n=1}
          \cos\frac{n\pi\theta}{\phi}\cos\frac{n\pi\theta_0}{\phi}
          \sum^{\infty}_{m=1}
          2D\mathrm{e}^{-\alpha_{mn}^2Dt}
          \frac{J_{\frac{n\pi}{\phi}}(r_0\alpha_{mn})
                J_{\frac{n\pi}{\phi}}(r\alpha_{mn})}
               {a^2\{J'_{\frac{n\pi}{\phi}}(a\alpha_{mn})\}^2}\nonumber\\
    v &=& \frac{2}{\phi a^2} \sum^{\infty}_{m=1}
          \mathrm{e}^{-\alpha_{m0}^2Dt}
          \frac{J_0(r_0\alpha_{m0})
                J_0(r\alpha_{m0})}
               {\{J_1(a\alpha_{m0})\}^2}+\nonumber\\
      & & \frac{4}{\phi a^2}\sum^{\infty}_{n=1}
          \cos\frac{n\pi\theta}{\phi}\cos\frac{n\pi\theta_0}{\phi}
          \sum^{\infty}_{m=1}
          \mathrm{e}^{-\alpha_{mn}^2Dt}
          \frac{J_{\frac{n\pi}{\phi}}(r_0\alpha_{mn})
                J_{\frac{n\pi}{\phi}}(r\alpha_{mn})}
               {\{J'_{\frac{n\pi}{\phi}}(a\alpha_{mn})\}^2}
\end{eqnarray}
this is the Green's function of this system.

\subsection{applying to diffusion on the lateral surface of a cone}
\label{sec:applying-surface-of-cone}

When the initial position is $\theta_0 = \phi / 2$, the boundary condition
of $\theta$ is equivalent to the condision that positions $\theta = 0$ and
$\theta = \phi$ are connected. This condition can be regarded as diffusion
equation on the lateral surface of the cone that has apical angle $\phi$ 
as development view.

If $\theta_0 = \phi / 2$, the Green's funciton become
\begin{eqnarray}
    G &=& \frac{2}{\phi a^2} \sum^{\infty}_{m=1}
          \mathrm{e}^{-\alpha_{m0}^2Dt}
          \frac{J_0(r_0\alpha_{m0})
                J_0(r\alpha_{m0})}
               {\{J_1(a\alpha_{m0})\}^2}+\nonumber\\
      & & \frac{4}{\phi a^2}\sum^{\infty}_{n=1}
          \cos\frac{n\pi\theta}{\phi}\cos\frac{n\pi}{2}
          \sum^{\infty}_{m=1}
          \mathrm{e}^{-\alpha_{mn}^2Dt}
          \frac{J_{\frac{n\pi}{\phi}}(r_0\alpha_{mn})
                J_{\frac{n\pi}{\phi}}(r\alpha_{mn})}
               {\{J'_{\frac{n\pi}{\phi}}(a\alpha_{mn})\}^2}
\end{eqnarray}
with odd $n$, the second term become zero. So this is equal to
\begin{eqnarray}
    G &=& \frac{2}{\phi a^2} \sum^{\infty}_{m=1}
          \mathrm{e}^{-\alpha_{m0}^2Dt}
          \frac{J_0(r_0\alpha_{m0})
                J_0(r\alpha_{m0})}
               {\{J_1(a\alpha_{m0})\}^2}+\nonumber\\
      & & \frac{4}{\phi a^2}\sum^{\infty}_{n=1}
          \cos\frac{2n\pi\theta}{\phi}\cos n\pi
          \sum^{\infty}_{m=1}
          \mathrm{e}^{-\alpha_{mn}^2Dt}
          \frac{J_{\frac{2n\pi}{\phi}}(r_0\alpha_{mn})
                J_{\frac{2n\pi}{\phi}}(r\alpha_{mn})}
               {\{J'_{\frac{2n\pi}{\phi}}(a\alpha_{mn})\}^2}
\end{eqnarray}

If $\phi = 2\pi$, the Green's function should be equal to 2DAbs
that of initial position is $\pi$.2DRefWedgeAbs' Green's function in the case of $\phi = 2\pi$ is shown below.
\begin{eqnarray}
      & & \left.G_{RefWedgeAbs}\right|_{\phi = 2\pi}\nonumber \\
      &=& \frac{1}{\pi a^2} \sum^{\infty}_{m=1}
          \mathrm{e}^{-\alpha_{m0}^2Dt}
          \frac{J_0(r_0\alpha_{m0})
                J_0(r\alpha_{m0})}
               {\{J_1(a\alpha_{m0})\}^2}+\nonumber\\
      & & \frac{2}{\pi a^2}\sum^{\infty}_{n=1}
          \cos n\theta\cos n\pi
          \sum^{\infty}_{m=1}
          \mathrm{e}^{-\alpha_{mn}^2Dt}
          \frac{J_n(r_0\alpha_{mn})
                J_n(r\alpha_{mn})}
               {\{J'_n(a\alpha_{mn})\}^2}\nonumber 
\end{eqnarray}
That of 2DAbs is
\begin{eqnarray}
      & & \left.G_{Abs}\right|_{\theta_0 = \pi}\nonumber\\
      &=& \frac{1}{\pi a^2} \sum^{\infty}_{m=1}
          \mathrm{e}^{-\alpha_{m0}^2Dt}
          \frac{J_0(r_0\alpha_{m0})
                J_0(r\alpha_{m0})}
               {\{J_1(a\alpha_{m0})\}^2}+\nonumber\\
      & & \frac{2}{\pi a^2}\sum^{\infty}_{n=1}
          \cos n(\theta - \pi)
          \sum^{\infty}_{m=1}
          \mathrm{e}^{-\alpha_{mn}^2Dt}
          \frac{J_n(r_0\alpha_{mn})
                J_n(r\alpha_{mn})}
               {\{J'_n(a\alpha_{mn})\}^2} \nonumber \\
%       &=& \frac{1}{\pi a^2} \sum^{\infty}_{m=1}
%           \mathrm{e}^{-\alpha_{m0}^2Dt}
%           \frac{J_0(r_0\alpha_{m0})
%                 J_0(r\alpha_{m0})}
%                {\{J_1(a\alpha_{m0})\}^2}+\nonumber\\
%       & & \frac{2}{\pi a^2}\sum^{\infty}_{n=1}
%          (\cos n\theta\cos n\pi + \sin n\theta\sin n\pi)
%           \sum^{\infty}_{m=1}
%           \mathrm{e}^{-\alpha_{mn}^2Dt}
%           \frac{J_n(r_0\alpha_{mn})
%                 J_n(r\alpha_{mn})}
%                {\{J'_n(a\alpha_{mn})\}^2} \nonumber \\
      &=& \frac{1}{\pi a^2} \sum^{\infty}_{m=1}
          \mathrm{e}^{-\alpha_{m0}^2Dt}
          \frac{J_0(r_0\alpha_{m0})
                J_0(r\alpha_{m0})}
               {\{J_1(a\alpha_{m0})\}^2}+\nonumber\\
      & & \frac{2}{\pi a^2}\sum^{\infty}_{n=1}
          \cos n\theta\cos n\pi
          \sum^{\infty}_{m=1}
          \mathrm{e}^{-\alpha_{mn}^2Dt}
          \frac{J_n(r_0\alpha_{mn})
                J_n(r\alpha_{mn})}
               {\{J'_n(a\alpha_{mn})\}^2} \nonumber 
\end{eqnarray}
These are same.

\section{public member function}

\subsection{drawR(rnd, t) const}

drawR returns the distance between
the position of particle and the center of system in time t.
the return value $r$ satisfies
\begin{equation}
    \int^r_0 \int^{2\pi}_0 P(r', \theta', t) r'dr'd\theta' =
    \mathrm{p\_survival}(t) \times \mathrm{rnd}.
\end{equation}
p\_int\_r calculates the integration
$\int^r_0 \int^{2\pi}_0 P(r', \theta', t) r'dr'd\theta'$.

\subsection{drawTheta(rnd, r, t) const}

drawTheta returns the angle $\theta$ of the particle from initial position in time $t$ and radius $r$.

The return value $\theta$ satisfies the equation
\begin{equation}
    \cfrac{\int^\theta_0 P(r, \theta', t) d\theta'}
          {\int^{2\pi}_0 P(r, \theta', t) d\theta'} = \mathrm{rnd}
\end{equation}

this function should be called after drawR was called and
the value of $r$ determined.

p\_int\_theta calculate the integration
$\int^\theta_0 P(r, \theta', t) d\theta'$.
and p\_int\_2pi calculate the integration
$\int^\pi_0 P(r, \theta', t) d\theta'$.

\subsection{drawTime(rnd) const}

drawTime returns the time when the particle goes out of the boundary.
Return value $t$ satisfies $1 - \mathrm{p\_survival}(t) = \mathrm{rnd}$.

\subsection{p\_survival(t) const}

\begin{eqnarray}
    S(t) &=& \int^a_0 \int^{\phi}_0 P(r, \theta, t) rdrd\theta \nonumber \\
         &=& \int^a_0 \frac{2}{a^2} \sum^{\infty}_{m=1}
             \mathrm{e}^{-\alpha_{m0}^2Dt}
             \frac{J_0(r_0\alpha_{m0})
                   J_0(r\alpha_{m0})}
                  {\{J_1(a\alpha_{m0})\}^2}rdrd\theta+\nonumber\\
         & & \int^a_0\int^\phi_0
             \frac{4}{\phi a^2}\sum^{\infty}_{n=1}
             \cos\frac{2n\pi\theta}{\phi}\cos n\pi
             \sum^{\infty}_{m=1}
             \mathrm{e}^{-\alpha_{mn}^2Dt}
             \frac{J_{\frac{2n\pi}{\phi}}(r_0\alpha_{mn})
                   J_{\frac{2n\pi}{\phi}}(r\alpha_{mn})}
                  {\{J'_{\frac{2n\pi}{\phi}}(a\alpha_{mn})\}^2}
             rdrd\theta\nonumber\\
         &=& \frac{2}{a} \sum^{\infty}_{m=1}
             \frac{J_0(r_0\alpha_{m0})}{\alpha_{m0} J_1(a\alpha_{m0})}
             \mathrm{e}^{-\alpha^2_{m0}Dt} \nonumber
\end{eqnarray}

\subsection{p\_int\_r(r, t) const}

\begin{eqnarray}
    & & \int^r_0 \int^{2\pi}_0 P(r', \theta, t) r'dr'd\theta \nonumber \\
    &=& \int^r_0 \frac{2}{a^2} \sum^{\infty}_{m=1}
        \mathrm{e}^{-\alpha_{m0}^2Dt}
        \frac{J_0(r_0\alpha_{m0})
              J_0(r'\alpha_{m0})}
             {\{J_1(a\alpha_{m0})\}^2} r'dr'd\theta+\nonumber\\
    & & \int^r_0\int^\phi_0
        \frac{4}{\phi a^2}\sum^{\infty}_{n=1}
        \cos\frac{2n\pi\theta}{\phi}\cos n\pi
        \sum^{\infty}_{m=1}
        \mathrm{e}^{-\alpha_{mn}^2Dt}
        \frac{J_{\frac{2n\pi}{\phi}}(r_0\alpha_{mn})
              J_{\frac{2n\pi}{\phi}}(r'\alpha_{mn})}
             {\{J'_{\frac{2n\pi}{\phi}}(a\alpha_{mn})\}^2}
        r'dr'd\theta\nonumber\\
    &=& \frac{2r}{a^2}\sum^\infty_{m=1}
        \frac{J_1(r\alpha_{m0})J_0(r_0\alpha_{m0})}
             {\alpha_{m0}J^2_1(a\alpha_{m0})}
        \mathrm{e}^{-\alpha^2_{m0}Dt}\nonumber
\end{eqnarray}

\subsection{p\_int\_theta(r, $\theta$, t) const}

Note: It is required that the value of $r$ is already determined by using drawR when this function is called.

This function returns the value of cummurative probability in $[0,\theta)$ for certain value of r.

\begin{eqnarray}
    & & \int^\theta_0 P(r, \theta', t) d\theta' \nonumber\\
%     &=& \frac{2\theta}{\phi a^2} \sum^{\infty}_{m=1}
%         \frac{J_0(r\alpha_{m0}) J_0(r_0\alpha_{m0})}
%              {J^2_1(a\alpha_{m0})}
%         \mathrm{e}^{-\alpha^2_{m0}Dt} \nonumber \\
%     &+& \int^\theta_0
%         \frac{4}{\phi a^2}\sum^{\infty}_{n=1}
%         \cos\frac{2n\pi\theta'}{\phi}\cos n\pi
%         \sum^{\infty}_{m=1}
%         \mathrm{e}^{-\alpha_{mn}^2Dt}
%         \frac{J_{\frac{2n\pi}{\phi}}(r_0\alpha_{mn})
%               J_{\frac{2n\pi}{\phi}}(r\alpha_{mn})}
%              {\{J'_{\frac{2n\pi}{\phi}}(a\alpha_{mn})\}^2}
%         d\theta' \nonumber \\
%     &=& \frac{2\theta}{\phi a^2} \sum^{\infty}_{m=1}
%         \frac{J_0(r\alpha_{m0}) J_0(r_0\alpha_{m0})}
%              {J^2_1(a\alpha_{m0})}
%         \mathrm{e}^{-\alpha^2_{m0}Dt} \nonumber \\
%     &+& \frac{4}{\phi a^2}\sum^{\infty}_{n=1}
%         \frac{\phi}{2n\pi}\sin\frac{2n\pi\theta}{\phi}\cos n\pi
%         \sum^{\infty}_{m=1}
%         \mathrm{e}^{-\alpha_{mn}^2Dt}
%         \frac{J_{\frac{2n\pi}{\phi}}(r_0\alpha_{mn})
%               J_{\frac{2n\pi}{\phi}}(r\alpha_{mn})}
%              {\{J'_{\frac{2n\pi}{\phi}}(a\alpha_{mn})\}^2}
%         \nonumber\\
%     &=& \frac{2\theta}{\phi a^2} \sum^{\infty}_{m=1}
%         \frac{J_0(r\alpha_{m0}) J_0(r_0\alpha_{m0})}
%              {J^2_1(a\alpha_{m0})}
%         \mathrm{e}^{-\alpha^2_{m0}Dt} \nonumber \\
%     &+& \frac{2}{\pi a^2}\sum^{\infty}_{n=1}
%         \frac{(-1)^n}{n}\sin\frac{2n\pi\theta}{\phi}
%         \sum^{\infty}_{m=1}
%         \mathrm{e}^{-\alpha_{mn}^2Dt}
%         \frac{4J_{\frac{2n\pi}{\phi}}(r_0\alpha_{mn})
%               J_{\frac{2n\pi}{\phi}}(r\alpha_{mn})}
%              {(J_{\frac{2n\pi}{\phi} - 1}(a\alpha_{mn})
%                   - J_{\frac{2n\pi}{\phi} + 1}(a\alpha_{mn}))^2}
%         \nonumber\\
    &=& \frac{2\theta}{\phi a^2} \sum^{\infty}_{m=1}
        \frac{J_0(r\alpha_{m0}) J_0(r_0\alpha_{m0})}
             {J^2_1(a\alpha_{m0})}
        \mathrm{e}^{-\alpha^2_{m0}Dt} \nonumber \\
    &+& \frac{8}{\pi a^2}\sum^{\infty}_{n=1}
        \frac{(-1)^n}{n}\sin\frac{2n\pi\theta}{\phi}
        \sum^{\infty}_{m=1}
        \mathrm{e}^{-\alpha_{mn}^2Dt}
        \frac{J_{\frac{2n\pi}{\phi}}(r_0\alpha_{mn})
              J_{\frac{2n\pi}{\phi}}(r\alpha_{mn})}
             {(J_{\frac{2n\pi}{\phi} - 1}(a\alpha_{mn})
             - J_{\frac{2n\pi}{\phi} + 1}(a\alpha_{mn}))^2}
        \nonumber
\end{eqnarray}

    argument of this funciton is in range (0, phi/2).
\begin{eqnarray}
    & & \int^{\frac{\phi}{2} + \theta}_{\frac{\phi}{2} - \theta}
       P(r, \theta', t) d\theta' \nonumber\\
    &=& \frac{4\theta}{\phi a^2} \sum^{\infty}_{m=1}
        \frac{J_0(r\alpha_{m0}) J_0(r_0\alpha_{m0})}
             {J^2_1(a\alpha_{m0})}
        \mathrm{e}^{-\alpha^2_{m0}Dt} \nonumber \\
    &+& \int^{\frac{\phi}{2} + \theta}_{\frac{\phi}{2} - \theta}
        \frac{4}{\phi a^2}\sum^{\infty}_{n=1}
        (-1)^n\cos\frac{2n\pi\theta'}{\phi}
        \sum^{\infty}_{m=1}
        \mathrm{e}^{-\alpha_{mn}^2Dt}
        \frac{J_{\frac{2n\pi}{\phi}}(r_0\alpha_{mn})
              J_{\frac{2n\pi}{\phi}}(r\alpha_{mn})}
             {\{J'_{\frac{2n\pi}{\phi}}(a\alpha_{mn})\}^2}
        d\theta' \nonumber \\
    &=& \frac{4\theta}{\phi a^2} \sum^{\infty}_{m=1}
        \frac{J_0(r\alpha_{m0}) J_0(r_0\alpha_{m0})}
             {J^2_1(a\alpha_{m0})}
        \mathrm{e}^{-\alpha^2_{m0}Dt} \nonumber \\
    &+& \frac{4}{\phi a^2}\sum^{\infty}_{n=1}
        (-1)^n\frac{(-1)^n\phi}{n\pi}\sin\frac{2n\pi\theta}{\phi}
        \sum^{\infty}_{m=1}
        \mathrm{e}^{-\alpha_{mn}^2Dt}
        \frac{J_{\frac{2n\pi}{\phi}}(r_0\alpha_{mn})
              J_{\frac{2n\pi}{\phi}}(r\alpha_{mn})}
             {\{J'_{\frac{2n\pi}{\phi}}(a\alpha_{mn})\}^2}
             \nonumber\\
    &=& \frac{4\theta}{\phi a^2} \sum^{\infty}_{m=1}
        \frac{J_0(r\alpha_{m0}) J_0(r_0\alpha_{m0})}
             {J^2_1(a\alpha_{m0})}
        \mathrm{e}^{-\alpha^2_{m0}Dt} \nonumber \\
    &+& \frac{4}{\pi a^2}\sum^{\infty}_{n=1}
        \frac{1}{n}\sin\frac{2n\pi\theta}{\phi}
        \sum^{\infty}_{m=1}
        \mathrm{e}^{-\alpha_{mn}^2Dt}
        \frac{J_{\frac{2n\pi}{\phi}}(r_0\alpha_{mn})
              J_{\frac{2n\pi}{\phi}}(r\alpha_{mn})}
             {\{J'_{\frac{2n\pi}{\phi}}(a\alpha_{mn})\}^2}\nonumber\\
    &=& \frac{4\theta}{\phi a^2} \sum^{\infty}_{m=1}
        \frac{J_0(r\alpha_{m0}) J_0(r_0\alpha_{m0})}
             {J^2_1(a\alpha_{m0})}
        \mathrm{e}^{-\alpha^2_{m0}Dt} \nonumber \\
    &+& \frac{16}{\pi a^2}\sum^{\infty}_{n=1}
        \frac{1}{n}\sin\frac{2n\pi\theta}{\phi}
        \sum^{\infty}_{m=1}
        \mathrm{e}^{-\alpha_{mn}^2Dt}
        \frac{J_{\frac{2n\pi}{\phi}}(r_0\alpha_{mn})
              J_{\frac{2n\pi}{\phi}}(r\alpha_{mn})}
             {(J_{\frac{2n\pi}{\phi} - 1}(a\alpha_{mn})
             - J_{\frac{2n\pi}{\phi} + 1}(a\alpha_{mn}))^2}
        \nonumber
%              {\{J'_{\frac{2n\pi}{\phi}}(a\alpha_{mn})\}^2}\nonumber
\end{eqnarray}

\subsection{p\_int\_phi(r, t) const}

Note: It is required that the value of $r$ is already determined by using drawR when this function is called.

This function returns the value of cummurative probability in $[0,\phi)$ for certain value of r.

\begin{eqnarray}
    & & \int^{\phi}_0 P(r, \theta, t) d\theta \nonumber\\
    &=& \frac{2}{a^2} \sum^{\infty}_{m=1}
        \frac{J_0(r\alpha_{m0}) J_0(r_0\alpha_{m0})}
             {J^2_1(a\alpha_{m0})}
        \mathrm{e}^{-\alpha^2_{m0}Dt} \nonumber \\
    &+& \int^{\phi}_0
        \frac{4}{\phi a^2}\sum^{\infty}_{n=1}
        \cos\frac{2n\pi\theta}{\phi}\cos n\pi
        \sum^{\infty}_{m=1}
        \mathrm{e}^{-\alpha_{mn}^2Dt}
        \frac{J_{\frac{2n\pi}{\phi}}(r_0\alpha_{mn})
              J_{\frac{2n\pi}{\phi}}(r\alpha_{mn})}
             {\{J'_{\frac{2n\pi}{\phi}}(a\alpha_{mn})\}^2}
        d\theta \nonumber \\
    &=& \frac{2}{a^2} \sum^{\infty}_{m=1}
        \frac{J_0(r\alpha_{m0}) J_0(r_0\alpha_{m0})}{J^2_1(a\alpha_{m0})}
        \mathrm{e}^{-\alpha^2_{m0}Dt} \nonumber
\end{eqnarray}

\subsection{dp\_int\_phi(t)}

This function returns the value of integral of $\frac{\partial P}{\partial r}$ in $[0,\phi)$ at $r = a$.

\begin{eqnarray}
    & & \frac{\partial P(r, \theta, t)}{\partial r} \nonumber\\
    &=& \frac{\partial}{\partial r}\left(
        \frac{2}{\phi a^2} \sum^{\infty}_{m=1}
        \mathrm{e}^{-\alpha_{m0}^2Dt}
        \frac{J_0(r_0\alpha_{m0})J_0(r\alpha_{m0})}
             {\{J_1(a\alpha_{m0})\}^2}\right) + \nonumber\\
    & & \frac{\partial}{\partial r}\left(
        \frac{16}{\phi a^2}\sum^{\infty}_{n=1}
        (-1)^n\cos\frac{2n\pi\theta}{\phi}
        \sum^{\infty}_{m=1}
        \mathrm{e}^{-\alpha_{mn}^2Dt}
        \frac{J_{\frac{2n\pi}{\phi}}(r_0\alpha_{mn})
              J_{\frac{2n\pi}{\phi}}(r  \alpha_{mn})}
             {(J_{\frac{2n\pi}{\phi} - 1}(a\alpha_{mn}) -
               J_{\frac{2n\pi}{\phi} + 1}(a\alpha_{mn}))^2}
        \right)
        \nonumber\\
    &=& \frac{2}{\phi a^2} \sum^{\infty}_{m=1}
        \mathrm{e}^{-\alpha_{m0}^2Dt}
        \frac{J_0(r_0\alpha_{m0})}{\{J_1(a\alpha_{m0})\}^2}
        \frac{\partial J_0(r\alpha_{m0})}{\partial r} + \nonumber\\
    & & \frac{16}{\phi a^2}\sum^{\infty}_{n=1}
        (-1)^n\cos\frac{2n\pi\theta}{\phi}
        \sum^{\infty}_{m=1}
        \frac{J_{\frac{2n\pi}{\phi}}(r_0\alpha_{mn})
              \mathrm{e}^{-\alpha_{mn}^2Dt}}
             {(J_{\frac{2n\pi}{\phi} - 1}(a\alpha_{mn}) -
               J_{\frac{2n\pi}{\phi} + 1}(a\alpha_{mn}))^2}
        \frac{\partial J_{\frac{2n\pi}{\phi}}(r  \alpha_{mn})}
        {\partial r}\nonumber \\
    &=& \frac{2}{\phi a^2} \sum^{\infty}_{m=1}
        \mathrm{e}^{-\alpha_{m0}^2Dt}
        \frac{J_0(r_0\alpha_{m0})}{\{J_1(a\alpha_{m0})\}^2}
        (-\alpha_{m0}J_1(r\alpha_{m0})) + \nonumber\\
    & & \frac{16}{\phi a^2}\sum^{\infty}_{n=1}
        (-1)^n\cos\frac{2n\pi\theta}{\phi}
        \sum^{\infty}_{m=1}
        \mathrm{e}^{-\alpha_{mn}^2Dt}
        \frac{\alpha_{mn}J_{\frac{2n\pi}{\phi}}(r_0\alpha_{mn})
             (J_{\frac{2n\pi}{\phi} - 1}(r\alpha_{mn}) -
              J_{\frac{2n\pi}{\phi} + 1}(r\alpha_{mn}))}
             {2(J_{\frac{2n\pi}{\phi} - 1}(a\alpha_{mn}) -
              J_{\frac{2n\pi}{\phi} + 1}(a\alpha_{mn}))^2}
        \nonumber
\end{eqnarray}
and substituting a into above expression
\begin{eqnarray}
    & & \left.\frac{\partial P(r, \theta, t)}{\partial r}\right|_{r = a} \nonumber\\
    &=& \frac{2}{\phi a^2} \sum^{\infty}_{m=1}
        \mathrm{e}^{-\alpha_{m0}^2Dt}
        \frac{-\alpha_{m0}J_0(r_0\alpha_{m0})}{J_1(a\alpha_{m0})} +\nonumber\\
    & & \frac{8}{\phi a^2}\sum^{\infty}_{n=1}
        (-1)^n\cos\frac{2n\pi\theta}{\phi}
        \sum^{\infty}_{m=1}
        \mathrm{e}^{-\alpha_{mn}^2Dt}
        \frac{\alpha_{mn}J_{\frac{2n\pi}{\phi}}(r_0\alpha_{mn})}
             {J_{\frac{2n\pi}{\phi} - 1}(a\alpha_{mn}) -
              J_{\frac{2n\pi}{\phi} + 1}(a\alpha_{mn})}
        \nonumber
\end{eqnarray}
so
\begin{eqnarray}
    & & \int^{\phi}_{0}
        \left.\frac{\partial P(r, \theta, t)}{\partial r}\right|_{r = a}d\theta
        \nonumber\\
    &=& \frac{2}{a^2} \sum^{\infty}_{m=1}
        \mathrm{e}^{-\alpha_{m0}^2Dt}
        \frac{-\alpha_{m0}J_0(r_0\alpha_{m0})}{J_1(a\alpha_{m0})} +\nonumber\\
    & & \int^{\phi}_{0}
        \frac{8}{\phi a^2}\sum^{\infty}_{n=1}
        (-1)^n\cos\frac{2n\pi\theta}{\phi}
        \sum^{\infty}_{m=1}
        \mathrm{e}^{-\alpha_{mn}^2Dt}
        \frac{\alpha_{mn}J_{\frac{2n\pi}{\phi}}(r_0\alpha_{mn})}
             {J_{\frac{2n\pi}{\phi} - 1}(a\alpha_{mn}) -
              J_{\frac{2n\pi}{\phi} + 1}(a\alpha_{mn})}
        d\theta\nonumber\\
    &=& \frac{2}{a^2} \sum^{\infty}_{m=1}
        \mathrm{e}^{-\alpha_{m0}^2Dt}
        \frac{-\alpha_{m0}J_0(r_0\alpha_{m0})}{J_1(a\alpha_{m0})}\nonumber
\end{eqnarray}

\subsection{dp\_int\_theta(theta, t)}

This function returns the value of integral of $\frac{\partial P}{\partial r}$
from $\frac{\phi}{2} - \theta$ to $\frac{\phi}{2} + \theta)$ at $r = a$.
\begin{eqnarray}
    & & \int^{\frac{\phi}{2} + \theta}_{\frac{\phi}{2} - \theta}
        \left.\frac{\partial P(r, \theta, t)}{\partial r}\right|_{r = a}d\theta
        \nonumber\\
    &=& \frac{4\theta}{\phi a^2} \sum^{\infty}_{m=1}
        \mathrm{e}^{-\alpha_{m0}^2Dt}
        \frac{-\alpha_{m0}J_0(r_0\alpha_{m0})}{J_1(a\alpha_{m0})} +\nonumber\\
    & & \int^{\frac{\phi}{2} + \theta}_{\frac{\phi}{2} - \theta}
        \frac{8}{\phi a^2}\sum^{\infty}_{n=1}
        (-1)^n\cos\frac{2n\pi\theta}{\phi}
        \sum^{\infty}_{m=1}
        \mathrm{e}^{-\alpha_{mn}^2Dt}
        \frac{\alpha_{mn}J_{\frac{2n\pi}{\phi}}(r_0\alpha_{mn})}
             {J_{\frac{2n\pi}{\phi} - 1}(a\alpha_{mn}) -
              J_{\frac{2n\pi}{\phi} + 1}(a\alpha_{mn})}
        d\theta\nonumber\\
    &=& \frac{4\theta}{\phi a^2} \sum^{\infty}_{m=1}
        \mathrm{e}^{-\alpha_{m0}^2Dt}
        \frac{-\alpha_{m0}J_0(r_0\alpha_{m0})}{J_1(a\alpha_{m0})} +\nonumber\\
    & & \frac{8}{\phi a^2}\sum^{\infty}_{n=1}
        (-1)^n
        \left[\frac{\phi}{2n\pi}\sin\frac{2n\pi\theta}{\phi}
        \right]^{\frac{\phi}{2} + \theta}_{\frac{\phi}{2} - \theta}
        \sum^{\infty}_{m=1}
        \mathrm{e}^{-\alpha_{mn}^2Dt}
        \frac{\alpha_{mn}J_{\frac{2n\pi}{\phi}}(r_0\alpha_{mn})}
             {J_{\frac{2n\pi}{\phi} - 1}(a\alpha_{mn}) -
              J_{\frac{2n\pi}{\phi} + 1}(a\alpha_{mn})}
        \nonumber\\
    &=& \frac{4\theta}{\phi a^2} \sum^{\infty}_{m=1}
        \mathrm{e}^{-\alpha_{m0}^2Dt}
        \frac{-\alpha_{m0}J_0(r_0\alpha_{m0})}{J_1(a\alpha_{m0})} +\nonumber\\
    & & \frac{8}{\phi a^2}\sum^{\infty}_{n=1}(-1)^n
        \frac{(-1)^n\phi}{n\pi}\sin\frac{2n\pi\theta}{\phi}
        \sum^{\infty}_{m=1}\mathrm{e}^{-\alpha_{mn}^2Dt}
        \frac{\alpha_{mn}J_{\frac{2n\pi}{\phi}}(r_0\alpha_{mn})}
             {J_{\frac{2n\pi}{\phi} - 1}(a\alpha_{mn}) -
              J_{\frac{2n\pi}{\phi} + 1}(a\alpha_{mn})}
        \nonumber\\
    &=& \frac{4\theta}{\phi a^2} \sum^{\infty}_{m=1}
        \mathrm{e}^{-\alpha_{m0}^2Dt}
        \frac{-\alpha_{m0}J_0(r_0\alpha_{m0})}{J_1(a\alpha_{m0})} +\nonumber\\
    & & \frac{8}{\pi a^2}\sum^{\infty}_{n=1}
        \frac{1}{n}\sin\frac{2n\pi\theta}{\phi}
        \sum^{\infty}_{m=1}\mathrm{e}^{-\alpha_{mn}^2Dt}
        \frac{\alpha_{mn}J_{\frac{2n\pi}{\phi}}(r_0\alpha_{mn})}
             {J_{\frac{2n\pi}{\phi} - 1}(a\alpha_{mn}) -
              J_{\frac{2n\pi}{\phi} + 1}(a\alpha_{mn})}
        \nonumber\\
\end{eqnarray}




\end{document}
