\documentclass{article}
\begin{document}

\title{Pair2DRadInf}
\maketitle
\section{infinite medium}

\begin{eqnarray}
    u(r,t|r',0)=\frac{1}{4\pi Dt}e^{-R^2/4Dt} \nonumber \\
    R^2=r^2+r'^2-2rr'\cos(\theta-\theta') \nonumber
\end{eqnarray}

by Laplace Transformation $\frac{1}{2t}e^{-x^2/4Dt}\rightarrow K_0(qx)$

\begin{equation}
    \bar{u}(r,s|r',0)=\frac{1}{2\pi D}K_0(qR)
\end{equation}
here, 
$q\equiv\sqrt{s/D}$

and using this
\begin{eqnarray}
    K_0(\eta R)&=&\sum_{n=-\infty}^{\infty}\cos n(\theta-\theta')I_n(\eta r)K_n(\eta r') \mathrm{when} r<r' \nonumber \\
    K_0(\eta R)&=&\sum_{n=-\infty}^{\infty}\cos n(\theta-\theta')I_n(\eta r')K_n(\eta r) \mathrm{when} r>r' 
\end{eqnarray}

transformed u can be written as

\begin{equation}
    \bar{u}(r,s|r',0)=\frac{1}{2\pi D}\sum_{n=-\infty}^{\infty}\cos n(\theta-\theta')I_n(qr)K_n(qr'), when r<r'
\end{equation}
here, let $\theta'=0$, 

\begin{equation}
\bar{u}(r,s|r',0)=\frac{1}{2\pi D}\sum_{n=0}^{\infty}\epsilon_n\cos n\theta I_n(qr)K_n(qr'), when r<r'
\end{equation}
here, $\epsilon_n \equiv 2-\delta(n)$

\section{boundary condision}

\begin{eqnarray}
    \frac{\partial^2 v}{\partial r^2}+\frac{1}{r}\frac{\partial v}{\partial r}+\frac{1}{r^2}\frac{\partial^2 v}{\partial\theta^2}-\frac{1}{D}\frac{\partial v}{\partial t}=0 \nonumber \\
    v(r,0)=\delta(r-r')\delta(\theta-\theta') \nonumber \\
    \left.\frac{\partial v(r,t)}{\partial r}\right|_{r=a}=hv(a,t), a<r' \nonumber
\end{eqnarray}

$v(r,t)$ has a finit value in $r\rightarrow\infty$. 

so let $v = u+w$, u and w satisfies below.
\begin{eqnarray}
    \frac{\partial^2 u}{\partial r^2}+\frac{1}{r}\frac{\partial u}{\partial r}+\frac{1}{r^2}\frac{\partial^2 u}{\partial\theta^2}-\frac{1}{D}\frac{\partial u}{\partial t}&=&0 \\
    \frac{\partial^2 w}{\partial r^2}+\frac{1}{r}\frac{\partial w}{\partial r}+\frac{1}{r^2}\frac{\partial^2 w}{\partial\theta^2}-\frac{1}{D}\frac{\partial w}{\partial t}&=&0 \\
    u(r,0)=\delta(r-r')\delta(\theta-\theta'), w(r,0)&=&0 \\
    \left.\frac{\partial (u+w)(r,t)}{\partial r}\right|_{r=a}&=&h(u+w)(a,t),\ a<r'
\end{eqnarray}

and let u is a solution in the case of infinite medium. then, we should find w that makes v to satisfy the boundary conditions.

\begin{eqnarray}
    \frac{\partial^2 w}{\partial r^2}+\frac{1}{r}\frac{\partial w}{\partial r}+\frac{1}{r^2}\frac{\partial^2 w}{\partial\theta^2}-\frac{1}{D}\frac{\partial w}{\partial t}=0 \\
    w(r,0)=0\\
    \left.\frac{\partial (u+w)(r,t)}{\partial r}\right|_{r=a}=h(u+w)(a,t) a<r'
\end{eqnarray}

\subsection{obtaining the solution}

\begin{equation}
    \frac{\partial^2 w}{\partial r^2}+\frac{1}{r}\frac{\partial w}{\partial r}+\frac{1}{r^2}\frac{\partial^2 w}{\partial\theta^2}-\frac{1}{D}\frac{\partial w}{\partial t}=0
\end{equation}

using $q\equiv\sqrt{s/D}$

\begin{equation}
    \frac{\partial^2\bar{w}}{\partial r^2}+\frac{1}{r}\frac{\partial\bar{w}}{\partial r}+\frac{1}{r^2}\frac{\partial^2\bar{w}}{\partial\theta^2}-q^2\bar{w}=0
\end{equation}

separating valuables like 
$\bar{w}(r,\theta,q)\equiv\bar{w}_r(r,q)\bar{w}_\theta(\theta,q)$, 

\begin{eqnarray}
    \frac{\partial^2\bar{w}_r}{\partial r^2}\bar{w}_\theta+\frac{1}{r}\frac{\partial\bar{w}_r}{\partial r}\bar{w}_\theta+\frac{1}{r^2}\frac{\partial^2\bar{w}_\theta}{\partial\theta^2}\bar{w}_r-q^2\bar{w}_r\bar{w}_\theta=0 \\
    r^2\frac{\partial^2\bar{w}_r}{\partial r^2}/\bar{w}_r+r\frac{\partial\bar{w}_r}{\partial r}/\bar{w}_r+\frac{\partial^2\bar{w}_\theta}{\partial\theta^2}/\bar{w}_\theta-q^2r^2=0
\end{eqnarray}

using
$-n^2$
\begin{eqnarray}
    r^2\frac{\partial^2\bar{w}_r}{\partial r^2}+r\frac{\partial\bar{w}_r}{\partial r}-\left(n^2+q^2r^2\right)\bar{w}_r=0 \\
    \frac{\partial^2\bar{w}_\theta}{\partial\theta^2}=-n^2\bar{w}_\theta .
\end{eqnarray}

generally, w can be written as
\begin{equation}
    \bar{w}_r=A_n I_n(qr) + B_n K_n(qr), \bar{w}_\theta=C_n \cos n\theta + D_n \sin n\theta,
\end{equation}

so

\begin{equation}
  \bar{w}=\sum_{n=0}^{\infty}\left\{A_n I_n(qr) + B_n K_n(qr)\right\}\left(C_n \cos n\theta + D_n \sin n\theta\right) .
\end{equation}

from first boundary condition, $r \to \infty v \to 0$,
\begin{equation}
    \bar{w}=\sum_{n=0}^{\infty}B_n K_n(qr)\cos n\theta
\end{equation}

now, let
$B_n\rightarrow\epsilon_nB_n$, then v in r<r' become

\begin{equation}
  \bar{v}=\bar{u}+\bar{v}=\frac{1}{2\pi D}\sum_{n=0}^{\infty}\epsilon_n\cos n\theta I_n(qr)K_n(qr')+2\pi D\epsilon_nB_n K_n(qr)\cos n\theta=\frac{1}{2\pi D}\sum_{n=0}^{\infty}\epsilon_n\cos n\theta \left[I_n(qr)K_n(qr')+2\pi DB_n K_n(qr)\right] .
\end{equation}

and differenciation by r is
\begin{eqnarray}
  & & \frac{\partial \bar{v}}{\partial r}=\frac{1}{2\pi D}\sum_{n=0}^{\infty}\epsilon_n\cos n\theta \left[qI'_n(qr)K_n(qr')+2\pi DB_n qK'_n(qr)\right] , \\
  &=&\frac{1}{2\pi D}\sum_{n=0}^{\infty}\epsilon_n\cos n\theta \left[q\left\{I_{n+1}(qr)+\frac{n}{qr}I_n(qr)\right\}K_n(qr')+2\pi DB_n q\left\{-K_{n+1}(qr)+\frac{n}{qr}K_n(qr)\right\}\right] ,
\end{eqnarray}

from the boundary condition r=a<r'
\begin{eqnarray}
  0=\left.\frac{\partial\bar{v}}{\partial r}-h\bar{v}\right|_{r=a}=\frac{1}{2\pi D}\sum_{n=0}^{\infty}\epsilon_n\cos n\theta \left[\left\{qI_{n+1}(qa)+\frac{n}{a}I_n(qa)\right\}K_n(qr')+2\pi DB_n \left\{-qK_{n+1}(qa)+\frac{n}{a}K_n(qa)\right\}\right] ... \\
  -\frac{h}{2\pi D}\sum_{n=0}^{\infty}\epsilon_n\cos n\theta \left[I_n(qa)K_n(qr')+2\pi DB_n K_n(qa)\right] .
\end{eqnarray}

to satisfy this in arbitrary argument, 

\begin{eqnarray}
  \left\{qI_{n+1}(qa)+\left(\frac{n}{a}-h\right)I_n(qa)\right\}K_n(qr')+2\pi DB_n \left\{-qK_{n+1}(qa)+\left(\frac{n}{a}-h\right)K_n(qa)\right\}=0 , \\
  2\pi DB_n=\frac{-\left(ah-n\right)I_n(qa)+qaI_{n+1}(qa)}{\left(ah-n\right)K_n(qa)+qaK_{n+1}(qa)}K_n(qr') .
\end{eqnarray}

so $\bar{v}$ is

\begin{equation}
  \bar{v}=\frac{1}{2\pi D}\sum_{n=0}^{\infty}\epsilon_n\cos n\theta\times K_n(qr') \left[\frac{\left\{\left(ah-n\right)K_n(qa)+qaK_{n+1}(qa)\right\}I_n(qr)+K_n(qr)\left\{-\left(ah-n\right)I_n(qa)+qaI_{n+1}(qa)\right\}}{\left(ah-n\right)K_n(qa)+qaK_{n+1}(qa)}\right] .
\end{equation}

\subsection{inverse laplace transformation}

consider second Bromwich path. "Conduction of Heat in Solids", p. 393. FIG. 40.

\begin{equation}
  v=\frac{1}{2\pi i}\int_{AB}\bar{v}e^{pt}dp=-\frac{1}{2\pi i}\int_{FE+ED+DC}\bar{v}e^{pt}dp 
\end{equation}

first, let path FE is $p=u^2De^{\pi i}$, $u \rightarrow 0$. then, $q\equiv\sqrt{\frac{p}{D}}$ and $q=ue^{\frac{1}{2}\pi i}$

\begin{equation}
  \frac{1}{2\pi i}\int_{FE}\bar{v}e^{pt}dp=\frac{1}{2\pi i}\int_{\infty}^{0}\bar{v}e^{-u^2Dt}\left(-2uD\right)du=\frac{D}{\pi i}\int_{0}^{\infty}\bar{v}ue^{-u^2Dt}du .
\end{equation}

generally, mopdified Bessel function satisfies

\begin{eqnarray}
  K_n(ze^{\pm\frac{1}{2}\pi i})=\pm\frac{1}{2}\pi ie^{\mp\frac{1}{2}n\pi i}\left[-J_n(z)\pm iY_n(z)\right] ,\\
  I_n(ze^{\pm\frac{1}{2}\pi i})=e^{\pm\frac{1}{2}n\pi i}J_n(z) ,
\end{eqnarray}

therefore, using this,
\begin{eqnarray}
    K_n(ze^{\frac{1}{2}\pi i})=\frac{1}{2}\pi ie^{-\frac{1}{2}n\pi i}\left[-J_n(z)+ iY_n(z)\right],\,I_n(ze^{\frac{1}{2}\pi i})=e^{\frac{1}{2}n\pi i}J_n(z) , \\
    trans = {BesselK[n_, q x_] -> 1/2 Pi  I Exp[-Pi (n)/2 I] (-BesselJ[n, u x] + I BesselY[n, u x]), BesselI[n_, q x_] -> Exp[ Pi (n)/2 I] BesselJ[n, u x], q -> I u}
\end{eqnarray}

the path FE becomes
\begin{equation}
    \bar{v}_{FE}=\frac{1}{4D}\sum_{n=0}^{\infty}\epsilon_n\cos n\theta\left[\left\{P_nJ_n(ur')+Y_n(ur')Q_n\right\}+i\left\{-P_nY_n(ur')+J_n(ur')Q_n\right\}\right]\frac{-Q_nJ_n(ur)+Y_n(ur)P_n}{P_n^2+Q_n^2} .
\end{equation}

here, $P_n=(ah-n)J_n(ua)+uaJ_{n+1}(ua)$ and $Q_n=(ah-n)Y_n(ua)+uaY_{n+1}(ua)$.

in other hand, let the path DC is $p=u^2De^{-\pi i}$, $u:\,0\rightarrow\infty$. therefore, $q=ue^{-\frac{1}{2}\pi i}$

\begin{equation}
  K_n(ze^{-\frac{1}{2}\pi i})=-\frac{1}{2}\pi ie^{\frac{1}{2}n\pi i}\left[-J_n(z)- iY_n(z)\right],\,I_n(ze^{-\frac{1}{2}\pi i})=e^{-\frac{1}{2}n\pi i}J_n(z) ,
trans = {BesselK[n_,q x_] -> -1/2 Pi I Exp[Pi (n)/2 I] (-BesselJ[n, u x] - I BesselY[n, u x]), BesselI[n_, q x_] -> Exp[-Pi (n)/2 I] BesselJ[n, u x], q -> -I u}
\end{equation}

\begin{equation}
  \bar{v}_{DC}=\frac{1}{4D}\sum_{n=0}^{\infty}\epsilon_n\cos n\theta\left[\left\{P_nJ_n(ur')+Y_n(ur')Q_n\right\}-i\left\{-P_nY_n(ur')+J_n(ur')Q_n\right\}\right]\frac{-Q_nJ_n(ur)+Y_n(ur)P_n}{P_n^2+Q_n^2} .
\end{equation}

this is a conjugate of the case of path FE

therefore
\begin{equation}
  v=-\frac{1}{2\pi i}\int_{FE}\bar{v}e^{pt}dp-\frac{1}{2\pi i}\int_{DC}\bar{v}e^{pt}dp=-\frac{D}{\pi i}\int_{0}^{\infty}\bar{v}_{FE}ue^{-u^2Dt}du+\frac{D}{\pi i}\int_{0}^{\infty}\bar{v}_{DC}ue^{-u^2Dt}du=-\frac{2D}{\pi}\int_{0}^{\infty}\mathrm{Im}\left[\bar{v}_{FE}\right]ue^{-u^2Dt}du .
\end{equation}

so inverse fourier transformation becomes
\begin{eqnarray}
    v&=&\frac{1}{2\pi}\sum_{n=0}^{\infty}\epsilon_n\cos n\theta\int_{0}^{\infty}\frac{\bigl(J_n(ur')Q_n-P_nY_n(ur')\bigr)\bigl(J_n(ur)Q_n-P_nY_n(ur)\bigr)}{P_n^2+Q_n^2}ue^{-u^2Dt}du\\
     &=&\frac{1}{2\pi}\sum_{n=0}^{\infty}\epsilon_n\cos n\theta\int_{0}^{\infty}C_n(ur)C_n(ur')ue^{-u^2Dt}du ,
\end{eqnarray}

here, $C_n(z)=\frac{J_n(z)Q_n-P_nY_n(z)}{\left(P_n^2+Q_n^2\right)^{\frac{1}{2}}}$ .

unsolved problem

-そもそも, second Bromwich pathでの経路積分がゼロになる理由. 特異点の問題.

-経路EDの小円がゼロに収束する証明.

\end{document}
