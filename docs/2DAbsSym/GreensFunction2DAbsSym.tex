\documentclass{article}
\usepackage{indentfirst}
\usepackage{amsmath}
\usepackage{mathrsfs}

\begin{document}

\title{GreensFunction2DAbsSym}
\maketitle

\section{Overview}

System contains only one particle.
Initial position of the particle is center of the system, $r=0$
(Position is descrived by the cylindorical coordinate,
 $x = r\cos\theta, y = r\sin\theta$).

Thus the solution of the diffusion equation does not depend on $\theta$.

Boundary condition is Absorbance boundary $P(a,t) = 0$.
$P(r, t)$ is the probability dencity function.

\subsection{Green's Function}

In the case of absorbing boundary condition with cylindorical surface source
$r = r'$,
the Green's function can be descrived as below.

\begin{equation}
    G(r, t | r', \tau) = \frac{1}{\pi a^2}\sum^{\infty}_{n=1}
                         \frac{J_0(r\alpha_n)J_0(r'\alpha_n)}{J^2_1(a\alpha_n)}
                         \mathrm{e}^{-D\alpha^2_n (t - \tau)} \nonumber \\
\end{equation}
where $J_0$ is 1st kind Bessel's function
and $\alpha_n$ causes $J_0$ to be zero that is $J_0(\alpha_n) = 0$.

and the initial condition $F$ is

\begin{equation}
    F(r) = \delta(r)
\end{equation}
at $t=0$.

thus the probability function is

\begin{eqnarray}
    P &=& \int^a_0 \int^{2\pi}_0 F(r') G(r, t | r', 0) r'dr'd\theta'
          \nonumber \\
      &=& \frac{1}{\pi a^2}\sum^{\infty}_{n=1}
          \frac{J_0(r\alpha_n)}{J^2_1(a\alpha_n)}\mathrm{e}^{-D\alpha^2_n t}
          \nonumber
\end{eqnarray}

\subsection{derivation of the Green's function}

This derivation process is based on
"Conduction of Heat in Solids"(Carslaw and Jaeger, 1959, p.368, section 14.8).

the diffusion equation is
\begin{eqnarray}
    D\nabla^2 P(r, t) &=& \frac{\partial P(r,t)}{\partial t} \nonumber \\
    P(a,t) &=& 0 \nonumber \\
    P(r,0) &=& \delta(r)
\end{eqnarray}

First, consider the case that a line source of strength
$1/2\pi r'$ is destributed round the circle of radius $r'$ in infinite medium.
Then $u(r, t)$, the probability at the point $(r, t)$, is

\begin{eqnarray}
  u &=& \frac{1}{4\pi Dt}
    \int^{2\pi}_0 \mathrm{e}^{(r^2+r'^2-2rr'\cos(\theta'))/4Dt} d\theta'
        \nonumber \\
    &=& \frac{1}{4\pi Dt} \mathrm{e}^{(r^2+r'^2)/4Dt}
        \int^{2\pi}_0 \mathrm{e}^{rr'\cos(\theta')/4Dt} d\theta'
        \nonumber \\
    &=& \frac{1}{4\pi Dt} \mathrm{e}^{(r^2+r'^2)/4Dt}
        I_0 \left( \frac{rr'}{2Dt} \right)
        \nonumber \\
\end{eqnarray}
and using Laplace transformation,

\[
    \mathscr{L}\left( \frac{1}{2t}\mathrm{e}^{(x^2+x'^2)/4Dt}
    I_k\left(\frac{xx'}{2Dt}\right) \right) (p)
    = \begin{cases}
          I_k(qx')K_k(qx) & (x > x') \\
          I_k(qx)K_k(qx') & (x' > x)
      \end{cases}
\]
where $q = \sqrt{\cfrac{p}{D}}$,

$\bar{u}$ can be obtained(k = 0 case).

\[
\bar{u} = \begin{cases}
    \cfrac{1}{2 \pi D} I_0(qr')K_0(qr) & (r > r') \\
    \cfrac{1}{2 \pi D} I_0(qr)K_0(qr') & (r' > r)
\end{cases}
\]

In order to satisfy boundary condition, let the solution $v$,
this can be written as the form $v = u + w$.

Here, $w$ is also the solution of the equation, so $\bar{w}$ satisfies the laplace transformed($t \rightarrow p$) equation,

\begin{equation}
    \frac{d^2 \bar{w}}{dr^2} + \frac{1}{r}\frac{d\bar{w}}{dr} - q^2 \bar{w} = 0
    \label{LTequation}
\end{equation}
where $q = \sqrt{\cfrac{p}{D}}$.\\

And $w$ is selected to make $v$ to satisfy the boundary condition.

the solution of (\ref{LTequation}) which has finite value at the origin
$r \rightarrow 0$ is $AI_0(qr)$ where $A$ is arbitrary constant.

Then the boundary condition can be written as

\begin{equation}
    AI_0(qa) + \frac{1}{2 \pi D} I_0(qr') K_0(qa) = 0
\end{equation}
so the value of A is

\begin{equation}
    A = - \frac{1}{2 \pi D} \frac{I_0(qr') K_0(qa)}{I_0(qa)}
\end{equation}
using this value, $\bar{v}$ can be written as

\begin{eqnarray}
    \bar{v} &=& \bar{u} + \bar{w} \nonumber \\
            &=& \frac{1}{2\pi D} I_0(qr')K_0(qr)
                - \frac{1}{2 \pi D} \frac{I_0(qr') K_0(qa)}{I_0(qa)}I_0(qr)
                \nonumber \\
            &=& \frac{I_0(qr')}{2 \pi D}
                \left( -\frac{K_0(qa)I_0(qr)}{I_0(qa)} + K_0(qr) \right)
                \nonumber \\
            &=& \frac{I_0(qr')}{2\pi D I_0(qa)}
                \left( -K_0(qa)I_0(qr) + I_0(qa)K_0(qr) \right)
\end{eqnarray}
in $r > r'$. In the case of $r < r'$, $\bar{v}$ can be obtained like this way.

\[
    \bar{v} = \begin{cases}
        \cfrac{I_0(qr')}{2 \pi D I_0(qa)}(I_0(qa)K_0(qr) - I_0(qr)K_0(qa)) & (r > r') \\
        \cfrac{I_0(qr)}{2 \pi D I_0(qa)}(I_0(qa)K_0(qr') - I_0(qr')K_0(qa)) & (r' > r)
    \end{cases}
\]

Inverse Laplace transformation can be calculated generally
by integral of a function of complex valuable.

\begin{equation}
    f(t) = \frac{1}{2\pi i} \lim_{T \to \infty} 
           \int^{\gamma + iT}_{\gamma - iT} \mathrm{e}^{pt}F(p)dp
\end{equation}
here, $\gamma$ is greater than the real part of all singularities of $F(q)$.
Therefore the integration is done along the vertical line $Re(p) = \gamma$
and all of the singularities of $\bar{v}$ exist in the left of this line.

So the inverse transformation of $\bar{v}$ is

\begin{eqnarray}
    & & \mathscr{L}^{-1} \left(\bar{v}(p)\right)(t) \nonumber \\
    &=& \frac{1}{2\pi i} \lim_{T \to \infty}
        \int^{\gamma + iT}_{\gamma - iT} \mathrm{e}^{pt} \bar{v}(p) dp \nonumber \\
    &=& \frac{1}{2\pi i} \lim_{T \to \infty} \int^{\gamma + iT}_{\gamma - iT}
        \mathrm{e}^{pt} \frac{I_0(qr')}{2\pi DI_0(qa)}
        \left(I_0(qa)K_0(qr) - I_0(qr)K_0(qa) \right) dp
\end{eqnarray}
make the contour to be closed and 
based on the residue-theorem, the integral is written as

\begin{equation}
    \oint \mathrm{e}^{pt} \bar{v}(p) dp \nonumber \\
    = 2\pi i\sum_m Res(\mathrm{e}^{pt} \bar{v}(p), \beta'_m)
\end{equation}
here, $\beta'_m$ is a pole of $\mathrm{e}^{pt} \bar{v}(p)$.
$\beta'_m$ satisfies $I_0\left(\sqrt{\frac{\beta'_m}{D}} a\right) = 0$.

Using this formula,
\begin{eqnarray}
    & & \mathscr{L}^{-1} \left(\bar{v}(p)\right)(t) \nonumber \\
    &=& \frac{1}{2\pi i} 2\pi i
        \sum_m Res(\mathrm{e}^{pt} \bar{v}(p), \beta'_m) \nonumber \\
    &=& \sum_m Res(\mathrm{e}^{pt} \bar{v}(p), \beta'_m)
\end{eqnarray}

here, using the transformation of valuables $p = -Dz^2$, $q = iz$, the integration become
\begin{equation}
    \oint \mathrm{e}^{-Dz^2t} \frac{I_0(izr')}{2\pi DI_0(iza)}
    \left(I_0(iza)K_0(izr) - I_0(izr)K_0(iza) \right) (-2Dz)dz
\end{equation}

From the nature of Bessel functions,

\begin{eqnarray}
    I_0(ix) &=& J_0(x) \nonumber \\
    K_0(ix) &=& \frac{-\pi i}{2} (J_0(x) - iY_0(x)) \nonumber
\end{eqnarray}
and the zero point of $I_0(\alpha')$, satisfies $J_0(\alpha' / i) = 0$.

So the values of residues become
\begin{eqnarray}
    & & Res(\mathrm{e}^{pt} \bar{v}(p), \alpha'_m) \nonumber \\
    &=& Res\left(\frac{\mathrm{e}^{pt}}{2\pi D}
        \frac{I_0(qr')}{I_0(qa)}
        \left(I_0(qa)K_0(qr) - I_0(qr)K_0(qa)\right), \alpha'_m \right) \nonumber \\
        &=& Res(-2Dz\frac{\mathrm{e}^{-z^2Dt}}{2\pi D} \frac{J_0(zr')}{J_0(za)} 
        (J_0(za)\cfrac{-\pi i}{2}\left(J_0(zr) - iY_0(zr) \right) \nonumber \\
    & & - J_0(zr)\cfrac{-\pi i}{2}\left(J_0(za) - iY_0(za) \right) ), \alpha_m )\nonumber \\
\end{eqnarray}
where, $\alpha'_m = \sqrt{\frac{\beta'_m}{D}}$.
And $\alpha_m$ satisfies $J_0(a\alpha_m) = 0$.

The value can be calculated as
\begin{eqnarray}
    & & Res(\bar{v}(z)\mathrm{e}^{-z^2Dt}) \nonumber \\
    &=& \lim_{z \to \alpha_m} -2Dz(z - \alpha_m)
        \frac{\mathrm{e}^{-z^2Dt}}{2\pi D} \frac{J_0(zr')}{J_0(za)} \nonumber \\
    & & \left(J_0(za)\cfrac{-\pi i}{2}\left(J_0(zr) - iY_0(zr) \right)
        - J_0(zr)\cfrac{-\pi i}{2}\left(J_0(za) - iY_0(za) \right) \right) \nonumber \\
    &=& \lim_{z \to \alpha_m} -2Dz(z - \alpha_m)
        \frac{-\mathrm{e}^{-z^2Dt}}{4D} \frac{J_0(zr')}{J_0(za)}
        \left(J_0(za)Y_0(zr) - J_0(zr)Y_0(za) \right) \nonumber \\
    &=& \lim_{z \to \alpha_m} -2Dz \frac{(z - \alpha_m)}{J_0(za)}
        \frac{-\mathrm{e}^{-z^2Dt}}{4D} J_0(zr')
        \left(J_0(za)Y_0(zr) - J_0(zr)Y_0(za) \right) \nonumber \\
    &=& \lim_{z \to \alpha_m} \frac{-\mathrm{e}^{-z^2Dt}}{4D}
        \frac{(z - \alpha_m)}{J_0(za)}
        (-2Dz)J_0(zr')
        \left(J_0(za)Y_0(zr) - J_0(zr)Y_0(za) \right) \nonumber
\end{eqnarray}

The value of first part becomes
\begin{equation}
    \lim_{z \to \alpha_m} \frac{-\mathrm{e}^{-z^2Dt}}{4D}
    = \frac{-\mathrm{e}^{-\alpha_{m}^{2}Dt}}{4D}
\end{equation}

From the definition of derivative, second part becomes
\begin{eqnarray}
        \lim_{z \to \alpha_m} \frac{(z - \alpha_m)}{J_0(za)}
    &=& \lim_{z \to \alpha_m} \frac{(za - a\alpha_m)}{a(J_0(za) - J_0(a\alpha_m))} \nonumber \\
    &=& \frac{1}{aJ'_0(a\alpha_m)}
\end{eqnarray}

The last part is 
\begin{eqnarray}
    & & \lim_{z \to \alpha_m} -2DzJ_0(zr')
        \left(J_0(za)Y_0(zr) - J_0(zr)Y_0(za) \right) \nonumber \\
    &=& \lim_{z \to \alpha_m} -2DzJ_0(zr')J_0(za)Y_0(zr) +
        2DzJ_0(zr')J_0(zr)Y_0(za) \nonumber \\
    &=& 2D\alpha_m J_0(\alpha_m r')J_0(\alpha_m r)Y_0(\alpha_m a) \nonumber \\
\end{eqnarray}

and $Y_n$ satisfies the equation below
\begin{eqnarray}
    J_n(z)Y'_n(z) - J'_n(z)Y_n(z) &=& \frac{2}{\pi z} \nonumber \\
    \pi zJ_n(z)Y'_n(z) - \pi zJ'_n(z)Y_n(z) &=& 2 \nonumber \\
    \pi zJ'_n(z)Y_n(z) &=& -2 + \pi zJ_n(z)Y'_n(z) \nonumber \\
    Y_n(z) &=& \frac{-2}{\pi zJ'_n(z)} + \frac{J_n(z)Y'_n(z)}{J'_n(z)}
\end{eqnarray}
then the value can be written as
\begin{equation}
    \alpha_mY_0(a\alpha_m) = \frac{-2}{a\pi J'_0(a\alpha_m)}
\end{equation}

So, 
\begin{eqnarray}
    & & 2D\alpha_mJ_0(\alpha_mr')J_0(\alpha_mr)Y_0(\alpha_ma) \nonumber \\
    &=& 2DJ_0(\alpha_mr')J_0(\alpha_mr)\frac{-2}{a\pi J'_0(a\alpha_m)}
\end{eqnarray}

Therefore, the values of the residues are
\begin{eqnarray}
    & & Res(\bar{v}(z)\mathrm{e}^{-z^2Dt}, \alpha_m) \nonumber \\
    &=& \lim_{z \to \alpha_m} \frac{-\mathrm{e}^{-z^2Dt}}{4D}
        \frac{(z - \alpha_m)}{J_0(za)}
        (-2Dz)J_0(zr')
        \left(J_0(za)Y_0(zr) - J_0(zr)Y_0(za) \right) \nonumber \\
    &=& \frac{-\mathrm{e}^{-\alpha_{m}^{2}Dt}}{4D}
        \frac{1}{aJ'_0(a\alpha_m)}
        2DJ_0(\alpha_mr')J_0(\alpha_mr)\frac{-2}{\pi J'_0(\alpha_m)} \nonumber \\
    &=& \frac{\mathrm{e}^{-\alpha_{m}^{2}Dt}}{a^2\pi}
        \frac{J_0(\alpha_mr')J_0(\alpha_mr)}{\{J'_0(\alpha_m)\}^2}
\end{eqnarray}

From these results, the inverse laplace transformation of $\bar{v}$ is

\begin{eqnarray}
  v &=& \mathscr{L}^{-1} \left(\bar{v}(p)\right)(t) \nonumber \\
    &=& \sum_{\alpha_m} \frac{\mathrm{e}^{-\alpha_{m}^{2}Dt}}{\pi a^2}
        \frac{J_0(\alpha_mr')J_0(\alpha_mr)}{\{J'_0(\alpha_m)\}^2}
\end{eqnarray}

As a conclusion,

\begin{eqnarray}
    v &=& \frac{1}{\pi a^2}
          \sum^\infty_{n=1} \frac{J_0(r\alpha_n)J_0(r'\alpha_n)}{J^2_1(a\alpha_n)}
                            \mathrm{e}^{-\alpha^2_n Dt}
\end{eqnarray}
where $J_0(\alpha_n) = 0$.

this is the Green's function of this system($\tau = 0$).

\subsection{solve the equation directory}

Otherwise, the equation can be solved directory by using separation of valiables.

First, the equation can be written as below.

\begin{eqnarray}
    P(r,t) &=& R(r)T(t) \nonumber \\
    \frac{1}{R}\left(\frac{\partial^2 R}{\partial r^2}
    + \frac{1}{r}\frac{\partial R}{\partial r}\right)
    &=& \frac{1}{DT} \frac{\partial T}{\partial t} = -\lambda^2 \nonumber
\end{eqnarray}

Equation of $R$ is Bessel's differencial equation, so the solution of $R$ is

\begin{equation}
    R(r) = AJ_0(\lambda r) + BY_0(\lambda r) \nonumber
\end{equation}

and the equation of $T$ is simply

\begin{equation}
    T(r) = C\mathrm{e}^{- \lambda^2 \alpha t} \nonumber
\end{equation}

where $A$, $B$, and $C$ is constant.

Therefore, the general solution is

\begin{equation}
    P(r, t) = (AJ_0(\lambda r) + BY_0(\lambda r) )
              C \mathrm{e}^{- \lambda^2 \alpha t} \nonumber
\end{equation}

and now, P has finite value in $r \rightarrow 0$ when $t>0$, so $B = 0$.

From boundary condition, 

\begin{eqnarray}
    P(a, t) &=& AJ_0(\lambda a) C \mathrm{e}^{- \lambda^2 \alpha t} \nonumber \\
            &=& 0 \nonumber
\end{eqnarray}

the exponential cannot be zero,
so $\lambda a$ should satisfy $J_0(\lambda a) = 0$.

Both of $A$ and $C$ are arbitrary constant, so let $AC = A$.

Therefore, 

\begin{equation}
    P(r, t) = AJ_0(\alpha r) \mathrm{e}^{-\alpha^2 Dt} \nonumber
\end{equation}

where $J_0(a\alpha) = 0$.

Let initial condition F(r). F can be written as sum of these solutions like,

\begin{eqnarray}
    F(r) = \sum_n A_n J_0(\alpha_n r) \nonumber
\end{eqnarray}

Bessel function is complete system, so F can be written as expansion.

\begin{eqnarray}
    & & \int^a_0 J_0(\alpha_n r) F(r) 2\pi rdr \nonumber \\
    &=& \int^a_0 \sum^\infty_{n=1} A_n J_0(\alpha_n r)J_0(\alpha_n r) 2\pi rdr
        \nonumber
\end{eqnarray}
thus
\begin {equation}
    A_n = \frac{1}{\pi a^2 J^2_1(\alpha_n)} \int^a_0 J_0(\alpha_n r)
          F(r) 2 \pi rdr \nonumber
\end{equation}

therefore the solution is

\begin{eqnarray}
    P(r,t) &=& \sum^\infty_{n=1} \mathrm{e}^{-\alpha_n Dt} 
               \int^a_0 \frac{J_0(\alpha_n r')J_0(\alpha_n r)}{\pi a^2 J^2_1(a\alpha)}
               F(r') r'dr' \nonumber \\
           &=& \int^a_0 F(r') \frac{1}{\pi a^2} \sum^\infty_{n=1}
               \frac{J_0(\alpha_n r')J_0(\alpha_n r)}{\pi a^2 J^2_1(a\alpha)}
               \mathrm{e}^{-\alpha^2 Dt} r' dr' \nonumber
\end{eqnarray}

so the Green's function is

\begin{equation}
    G(r, t) = \frac{1}{\pi a^2} \sum^\infty_{n=1}
              \frac{J_0(\alpha_n r')J_0(\alpha_n r)}{J^2_1(a\alpha)}
              \mathrm{e}^{-\alpha^2 Dt} \nonumber
\end{equation}

this is equal to the form described above.

\section{public member function}

\subsection{drawR(rnd, t) const}

\begin{table}[htb]
    \begin{tabular}{ll}
        rnd    & const double \\
        t      & const double \\
        return & const double
    \end{tabular}
\end{table}

drawR returns the distance between particle position in time t and initial position.

the return value r satisfies
$\int^r_0 P(r', t) r'dr' = \mathrm{p\_survival}(t) \times \mathrm{rnd}$.

p\_int\_r(r) calculate the integration$\int^r_0 P(r',t) r'dr'$.

\subsection{drawTime(rnd) const}

\begin{table}[htb]
    \begin{tabular}{ll}
        rnd     & const double \\
        return  & const double
    \end{tabular}
\end{table}

drawTime returns when the particle goes out of the boundary($ = a$).

the return value t satisfies $ 1 - \mathrm{p\_survival}(t) = \mathrm{rnd}$.

\subsection{p\_survival(t) const}

\begin{table}[htb]
    \begin{tabular}{ll}
        t       & const double\\
        return  & const double
    \end{tabular}
\end{table}

p\_survival returns the survival probability of the particle in the system at time t.

\begin{eqnarray}
    S(t) &=& \int^a_0 \int^{2\pi}_0 P(r,t) rdr \nonumber \\
         &=& \frac{2}{a}\sum^{\infty}_{n=1}\frac{1}{\alpha_n J_1(a\alpha_n)}
             \mathrm{e}^{-D\alpha^2_nt}\nonumber
\end{eqnarray}

derivation of the function is
\begin{eqnarray}
    \int^a_0 \int^{2\pi}_0 P(r, t) r dr
    &=&
    2\pi \int^a_0 \frac{1}{\pi a}
    \sum^{\infty}_{n=1}\frac{J_0(r\alpha_n)}{J^2_1(a\alpha_n)}
                       \mathrm{e}^{-D\alpha^2_nt} rdr \nonumber \\
    &=& \frac{2}{a}\sum^{\infty}_{n=1}
                       \frac{1}{J^2_1(a\alpha_n)}
                       \mathrm{-D\alpha^2_nt} \int^a_0 J_0(r\alpha_n) rdr
        \nonumber \\
    &=& \frac{2}{a}\sum^{\infty}_{n=1}
                       \frac{1}{J^2_1(a\alpha_n)}
                       \mathrm{e}^{-D\alpha^2_nt} \frac{aJ_1(a\alpha_n)}{\alpha_n}
        \nonumber \\
    &=& \frac{2}{a}\sum^{\infty}_{n=1}
                       \frac{1}{\alpha_n J_1(a\alpha_n)}
                       \mathrm{e}^{-D\alpha^2_nt} \nonumber
\end{eqnarray}

\subsection{p\_int\_r(r, t) const}

\begin{table}[htb]
    \begin{tabular}{ll}
        r       & const double\\
        t       & const double\\
        return  & const double
    \end{tabular}
\end{table}

p\_int\_r returns the cumulative distriburion function $[0,r)$.
integration can be obtained in the same way as the case of p\_survival.

\begin{eqnarray}
    \int^r_0 \int^{2\pi}_0 P(r',t) r'dr' =
    \frac{2}{a^2}\sum^{\infty}_{n=1}
        \frac{rJ_1(r\alpha_n)}{\alpha_n J^2_1(a\alpha_n)}
        \mathrm{e}^{-D\alpha^2_nt}\nonumber
\end{eqnarray}

\end{document}
