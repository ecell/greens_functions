\documentclass{article}
\usepackage{amsmath}
\usepackage{mathrsfs}

\begin{document}

\title{GreensFunction2DRefWedgeInf}
\author{Toru Niina}
\maketitle

\section{Overview}

System contains only one particle. 
 Position is described by cylindrical coordinate system.
 Initial position of the particle is $r = r_0$, $\theta = \theta_0$.

And boundary conditions are
\begin{eqnarray}
    P(a, \theta, t) &=& 0 \nonumber \\
    \left.\frac{\partial P}{\partial\theta}\right|_{\theta = 0} &=& 0 \nonumber \\
    \left.\frac{\partial P}{\partial\theta}\right|_{\theta = \phi} &=& 0
    \label{boundary-conditions}
\end{eqnarray}
where $0 < \phi < 2\pi$.

When the condition $\theta_0 = \frac{\phi}{2}$ is satisfied, 
 these setups become equivalent to the condition that
 the diffusion on the lateral surface of infinite-height cone.

\subsection{Green's Function}
\begin{eqnarray}
    G(r, \theta, t)
    &=& \frac{1}{2\phi Dt}\mathrm{e}^{-(r'^2 + r^2)/4Dt} \times\nonumber \\
    & & \Biggl(I_0\left(\frac{rr'}{2Dt}\right) + 
        2\sum^{\infty}_{n=1}\cos\frac{n\pi\theta}{\phi}\cos\frac{n\pi\theta_0}{\phi}
        I_{n\pi\phi}\left(\frac{rr'}{2Dt}\right) \Biggr)\label{greens-function}
\end{eqnarray}

\subsection{Derivation of the Green's function}
As a starting point, consider 2 dimentional diffusion equation in infinite space
and the solution of that.

\begin{eqnarray}
    \frac{\partial u}{\partial t} &=& D\nabla^2 u \\
    u(R, t) &=& \frac{1}{4\pi Dt}\mathrm{exp}\left(\frac{-R^2}{4Dt}\right)
\end{eqnarray}
here, $R$ is a distance from initial position $(x_0, y_0)$ to $(x, y)$, so
\begin{equation}
 R = \sqrt{(x-x_0)^2+(y-y_0)^2}.
\end{equation}

When the equation is considered in Cylindrical coordinate, the form of solution
 is same as above but the definition of distance $R$ become

\begin{equation}
    R = \sqrt{r^2+r_0^2+rr_0\cos(\theta - \theta_0)}.
\end{equation}

Let the solution of diffusion equation $v$ under the boundary conditions
 (\ref{boundary-conditions}) as
\begin{equation}
    v = u + w.
\end{equation}
here, $u$ corresponds to the solution of diffusion equation in the case of
 infinite space. $w$ satisfies the same equation and cancels the value of $u$ on
 the boundary.

To solve this, use Laplace transformation.
 The solution of diffusion equation in infinite space with transformation
 $(t\rightarrow p)$ is

\begin{equation} \label{Laplace-u-solution-infinite}
    \bar{u} = \frac{1}{2\pi D}K_0(qR)
\end{equation}
here, q is defined as $q = \sqrt{\cfrac{p}{D}}$ and $K_0$ means modified Bessel
 function of the second kind.

Now, it can be written that
\begin{equation}\label{expansion-K0}
    K_0(qR) = P.V.\int^{\infty i}_{-\infty i}
              \frac{\cos\nu (\pi - \theta + \theta_0)}{\sin\nu\pi}
              K_\nu(qr)I_\nu(qr_0)id\nu
\end{equation}
where $0 < \theta - \theta_0 < 2\pi$ and $r_0 < r$.
 Here, $P.V.$ means Cauthy's principal value of the origin.

Then $\bar{u}$ can be re-written as
\begin{equation}\label{expansion-bar-u}
    \bar{u} = \frac{i}{2\pi D}P.V.\int^{\infty i}_{-\infty i}
              \frac{\cos\nu (\pi - \theta + \theta_0)}{\sin\nu\pi}
              K_\nu(qr)I_\nu(qr_0)d\nu
\end{equation}
where $0 < \theta - \theta_0 < 2\pi$ and $r_0 < r$.

In order to cancel out the value of $\cfrac{\partial\bar{u}}{\partial\theta}$ on the boundary,
 where $\theta = \phi$ and $r_0 < r$, $\bar{w}$ should satisfy below.
\begin{eqnarray}
    \left.\frac{\partial\bar{w}}{\partial\theta}\right|_{\theta = \phi}
    &=& \frac{-i}{2\pi D}P.V.\int^{\infty i}_{-\infty i}
              \frac{\nu\sin\nu(\pi - \phi + \theta_0)}{\sin\nu\pi}
              K_\nu(qr)I_\nu(qr_0)d\nu\label{boundary-value-phi}\\
    \left.\frac{\partial\bar{w}}{\partial\theta}\right|_{\theta = 0}
    &=& \frac{-i}{2\pi D}P.V.\int^{\infty i}_{-\infty i}
              \frac{-\nu\sin\nu(\pi - \theta_0)}{\sin\nu\pi}
              K_\nu(qr)I_\nu(qr_0)d\nu \label{boundary-value-zero}
\end{eqnarray}

With the Laplace transformation $t\rightarrow p$, subsidiary equation become
\begin{eqnarray}
    D\left( \frac{\partial^2}{\partial r^2} +
            \frac{1}{r}\frac{\partial}{\partial r} +
            \frac{1}{r^2}\frac{\partial^2}{\partial \theta^2}
            \right)
    \bar{\Psi}
    &=&
    p\bar{\Psi} \nonumber \\
    r^2 \frac{\partial^2}{\partial r^2}\bar{\Psi} +
    r   \frac{\partial}{\partial r}\bar{\Psi} +
        \frac{\partial^2}{\partial \theta^2}\bar{\Psi} -
    r^2 \frac{p}{D}\bar{\Psi} &=& 0.
\end{eqnarray}

Assuming that the form of the solution $\Psi$ can be written as
 $R(r, t)\Theta(\theta)$ and $\Theta(\theta)$ satisfies the equation below, 
\begin{equation}
    \frac{d^2}{d\theta^2}\bar{\Theta} = -n^2\bar{\Theta}
\end{equation}

The subsidialy equation become the same form as Modified Bessel's equation.
\begin{eqnarray} \label{subsidiary-eq}
    r^2\frac{\partial^2}{\partial r^2}\bar{\Psi} +
    r\frac{\partial}{\partial r}\bar{\Psi} -(n^2 + r^2q^2)\bar{\Psi} &=& 0.
\end{eqnarray}
So the solution of (\ref{subsidiary-eq}) is a multipletation between
 a linear combination of Modified Bessel function, $I_n(qr), K_n(qr)$ and
 a linear combination of $\sin\theta$ and $\cos\theta$.

\begin{equation}\label{solution-of-subsidialy-equation}
    solution = \sum_n\Bigl( A\cos n\theta + B\sin n\theta\Bigr)
    \Bigl(CI_n(rq) + DK_n(rq)\Bigr)
\end{equation}

Here, this integral
\begin{equation}\label{general-integral-form-of-w}
P.V.\int^{\infty i}_{-\infty i}
    \frac{\cos\nu\pi}{\sin\nu\pi}
    (A\cos\nu(\phi - \theta) + B\cos\nu\theta)
    K_\nu(qr)id\nu
\end{equation}
% is evaluated as
% \begin{equation}
%     \sum_n\Bigl( (A\cos n\phi + B)\cos n\theta + A\sin n\phi\sin n\theta\Bigr)K_n(qr)
% \end{equation}
% So the form (\ref{general-integral-form-of-w})
satisfies (\ref{subsidiary-eq}).

Assuming that $\bar{w}$ is
\begin{equation}
    \bar{w}
  = \frac{-i}{2\pi D}P.V.\int^{\infty i}_{-\infty i}
    \frac{\cos\nu\pi}{\sin\nu\pi}
    (A\cos\nu(\phi - \theta) + B\cos\nu\theta)
    K_\nu(qr)d\nu
\end{equation}
and determine the constants $A$ and $B$. From boundary condition
(\ref{boundary-value-phi}) and (\ref{boundary-value-zero})
\begin{eqnarray}
%
% theta = phi case.
%
    \left.\frac{\partial\bar{w}}{\partial\theta}\right|_{\theta = \phi}
    &=& \frac{i}{2\pi D}P.V.\int^{\infty i}_{-\infty i}
    \frac{\cos\nu\pi}{\sin\nu\pi}
    \nu B\sin\nu\phi
    K_\nu(qr)d\nu \nonumber \\
    &=& \frac{-i}{2\pi D}P.V.\int^{\infty i}_{-\infty i}
        \frac{\nu\sin\nu(\pi - \phi + \theta_0)}{\sin\nu\pi}
        K_\nu(qr)I_\nu(qr_0)d\nu \nonumber \\
%
% theta = zero case.
%
    \left.\frac{\partial\bar{w}}{\partial\theta}\right|_{\theta = 0}
    &=& \frac{-i}{2\pi D}P.V.\int^{\infty i}_{-\infty i}
    \frac{\cos\nu\pi}{\sin\nu\pi}
    \nu A\sin\nu\phi
    K_\nu(qr)d\nu \nonumber \\
    &=& \frac{-i}{2\pi D}P.V.\int^{\infty i}_{-\infty i}
        \frac{-\nu\sin\nu(\pi - \theta_0)}{\sin\nu\pi}
        K_\nu(qr)I_\nu(qr_0)d\nu
\end{eqnarray}
thus $A$ and $B$ are evaluated as
\begin{eqnarray}
    A &=& \frac{-\sin\nu(\pi - \theta_0)}{\cos\nu\pi\sin\nu\phi}I_\nu(qr_0) \nonumber \\
    B &=& \frac{-\sin\nu(\pi-\phi+\theta_0)}{\cos\nu\pi\sin\nu\phi}I_\nu(qr_0) \nonumber \\
\end{eqnarray}
So $\bar{w}$ is
\begin{eqnarray}
% procedure of derivation
    \bar{w} &=& \frac{-i}{2\pi D}P.V.\int^{\infty i}_{-\infty i}
                \frac{\cos\nu\pi}{\sin\nu\pi}
                \left(\frac{-\sin\nu(\pi - \theta_0)}{\cos\nu\pi\sin\nu\phi}\cos\nu(\phi - \theta) +
                      \frac{-\sin\nu(\pi-\phi+\theta_0)}{\cos\nu\pi\sin\nu\phi}\cos\nu\theta\right)\times\nonumber \\
            & & K_\nu(qr)I_\nu(qr_0)d\nu \nonumber \\
            &=& \frac{i}{2\pi D}P.V.\int^{\infty i}_{-\infty i}
                \frac{1}{\sin\nu\pi\sin\nu\phi}
                \left(\sin\nu(\pi - \theta_0)\cos\nu(\phi - \theta) +
                      \sin\nu(\pi-\phi+\theta_0)\cos\nu\theta\right)\nonumber \\
            & & K_\nu(qr)I_\nu(qr_0)d\nu \nonumber \\
    \bar{w} &=& \frac{i}{2\pi D}P.V.\int^{\infty i}_{-\infty i}
                \frac{\sin\nu(\pi - \theta_0)\cos\nu(\phi - \theta) +
                      \sin\nu(\pi-\phi+\theta_0)\cos\nu\theta}
                {\sin\nu\pi\sin\nu\phi}\nonumber \\
            & & K_\nu(qr)I_\nu(qr_0)d\nu\label{bar-w}
\end{eqnarray}


Then $\bar{v}$ can be obtained by simply add (\ref{expansion-bar-u}) and
 (\ref{bar-w}). And then the integration path can be completed at the 
 origin.

\begin{eqnarray} 
% procedure of derivation
    \bar{v} &=& \bar{u} + \bar{w} \nonumber \\
            &=& \frac{i}{2\pi D}P.V.\int^{\infty i}_{-\infty i}
                \frac{\cos\nu (\pi - \theta + \theta_0)}{\sin\nu\pi}
                K_\nu(qr)I_\nu(qr_0)d\nu +\nonumber \\
            & & \frac{i}{2\pi D}P.V.\int^{\infty i}_{-\infty i}
                \frac{\sin\nu(\pi - \theta_0)\cos\nu(\phi - \theta) +
                      \sin\nu(\pi-\phi+\theta_0)\cos\nu\theta}
                {\sin\nu\pi\sin\nu\phi}\times\nonumber \\
            & & K_\nu(qr)I_\nu(qr_0)d\nu \nonumber \\
            &=& \frac{i}{2\pi D}\int^{\infty i}_{-\infty i}\left[
                \frac{\cos\nu (\pi - \theta + \theta_0)}{\sin\nu\pi}+
                \frac{\sin\nu(\pi - \theta_0)\cos\nu(\phi - \theta) +
                      \sin\nu(\pi-\phi+\theta_0)\cos\nu\theta}
                     {\sin\nu\pi\sin\nu\phi}\right]\times\nonumber \\
            & & K_\nu(qr)I_\nu(qr_0)d\nu \nonumber \\
            &=& \frac{i}{2\pi D}\int^{\infty i}_{-\infty i}\left[
                \frac{\sin\nu\phi\cos\nu (\pi - \theta + \theta_0) +
                    \sin\nu(\pi - \theta_0)\cos\nu(\phi - \theta) +
                    \sin\nu(\pi-\phi+\theta_0)\cos\nu\theta}
                    {\sin\nu\pi\sin\nu\phi}\right]\times\nonumber \\
            & & K_\nu(qr)I_\nu(qr_0)d\nu \nonumber
\end{eqnarray}
numerator
\begin{eqnarray}
            & & \sin\nu\phi\cos\nu (\pi - \theta + \theta_0) +
                \sin\nu(\pi - \theta_0)\cos\nu(\phi - \theta) +
                \sin\nu(\pi-\phi+\theta_0)\cos\nu\theta\nonumber \\
            &=& \sin\nu\phi\cos\nu\pi\cos\nu\theta\cos\nu\theta_0 + \sin\nu\phi\cos\nu\pi\sin\nu\theta\sin\nu\theta_0 \nonumber \\
            &+& \sin\nu\phi\sin\nu\pi\sin\nu\theta\cos\nu\theta_0 - \sin\nu\phi\sin\nu\pi\cos\nu\theta\sin\nu\theta_0 \nonumber \\
            &+& \cos\nu\phi\sin\nu\pi\cos\nu\theta\cos\nu\theta_0 + \sin\nu\phi\sin\nu\pi\cos\nu\theta\sin\nu\theta_0 \nonumber \\
            &-& \sin\nu\phi\cos\nu\pi\cos\nu\theta\cos\nu\theta_0 + \cos\nu\phi\cos\nu\pi\cos\nu\theta\sin\nu\theta_0 \nonumber \\
            &+& \cos\nu\phi\sin\nu\pi\cos\nu\theta\cos\nu\theta_0 - \cos\nu\phi\cos\nu\pi\cos\nu\theta\sin\nu\theta_0 \nonumber \\
            &+& \sin\nu\phi\sin\nu\pi\sin\nu\theta\cos\nu\theta_0 - \sin\nu\phi\cos\nu\pi\sin\nu\theta\sin\nu\theta_0 \nonumber \\
            &=& \sin\nu\phi\sin\nu\pi\sin\nu\theta\cos\nu\theta_0 + \sin\nu\phi\sin\nu\pi\sin\nu\theta\cos\nu\theta_0 \nonumber \\
            &+& \cos\nu\phi\sin\nu\pi\cos\nu\theta\cos\nu\theta_0 + \cos\nu\phi\sin\nu\pi\cos\nu\theta\cos\nu\theta_0 \nonumber \\
            &=& 2\sin\nu\pi\cos\nu\theta_0(\sin\nu\phi\sin\nu\theta+ \cos\nu\phi\cos\nu\theta)\nonumber \\
            &=& 2\sin\nu\pi\cos\nu\theta_0\cos\nu(\phi - \theta)\nonumber \\
\end{eqnarray}
\begin{eqnarray}
    \bar{v} &=& \frac{i}{2\pi D}P.V.\int^{\infty i}_{-\infty i}\left[
                \frac{2\sin\nu\pi\cos\nu\theta_0\cos\nu(\phi - \theta)}
                    {\sin\nu\pi\sin\nu\phi}\right] K_\nu(qr)I_\nu(qr_0)d\nu \nonumber \\
            &=& \frac{i}{\pi D}P.V.\int^{\infty i}_{-\infty i}\left[
                \frac{\cos\nu\theta_0\cos\nu(\phi - \theta)}{\sin\nu\phi}
                \right] K_\nu(qr)I_\nu(qr_0)d\nu \nonumber \\
            &=& \frac{i}{\pi D}P.V.\int^{\infty i}_{-\infty i}
                \frac{\cos\nu\theta_0\cos\nu(\phi - \theta)}{\sin\nu\phi}
                K_\nu(qr)I_\nu(qr_0)d\nu\label{bar-v}
\end{eqnarray}

% This integral is calculated by completing the path by a large semicircle in the
%  right half plane, and evaluating the residues at the poles $\sin\nu\phi = 0$, $\nu = n\pi / \phi$.

Now, let
\begin{equation}
    s = n\pi / \phi
\end{equation}
then the values of residues are
\begin{eqnarray}
    & & \frac{i}{\pi D}\lim_{\nu\to s}\left[
        (\nu - s)\frac{i}{2\pi D}
                 \frac{\cos\nu\theta_0\cos\nu(\phi - \theta)}{\sin\nu\phi}
                 K_\nu(qr)I_\nu(qr_0)d\nu
        \right] \nonumber \\
    &=& \frac{i}{\pi D}\lim_{\nu\to s}\left[
        \frac{\nu\phi - s\phi}{\phi\sin(\nu\phi-s\phi)}
        \cos\nu\theta_0\cos\nu(\phi - \theta)
        K_\nu(qr)I_\nu(qr_0)
        \right] \nonumber \\
    &=& \frac{i}{\pi D\phi}\cos s\theta_0\cos s(\phi - \theta)K_s(qr)I_s(qr_0) \nonumber\\
    &=& \frac{i}{\pi D\phi}\cos\frac{n\pi\theta_0}{\phi}\cos\frac{n\pi(\phi - \theta)}{\phi}
        K_s(qr)I_s(qr_0) \nonumber\\
    &=& \frac{i}{\pi D\phi}\cos\frac{n\pi\theta_0}{\phi}
        (\cos\frac{n\pi\phi}{\phi}\cos\frac{n\pi\theta}{\phi} + \sin\frac{n\pi\phi}{\phi}\sin\frac{n\pi\theta}{\phi})
        K_s(qr)I_s(qr_0) \nonumber\\
    &=& \frac{i}{\pi D\phi}\cos\frac{n\pi\theta_0}{\phi}\cos\frac{n\pi\theta}{\phi}K_{\frac{n\pi}{\phi}}(qr)I_{\frac{n\pi}{\phi}}(qr_0) \nonumber
\end{eqnarray}
so from residue theorem,
\begin{eqnarray}
%     &=& \frac{1}{D\phi}\sum^{\infty}_{n=0}
%                 \cos\frac{n\pi\theta}{\phi}\cos\frac{n\pi\theta_0}{\phi}
%                 K_{\frac{n\pi}{\phi}}(qr)I_{\frac{n\pi}{\phi}}(qr_0) \nonumber \\
    \bar{v} &=& \frac{1}{D\phi}\left[K_0(qr)I_0(qr_0) + 
                2\sum^{\infty}_{n=1}
                \cos\frac{n\pi\theta}{\phi}\cos\frac{n\pi\theta_0}{\phi}
                K_{\frac{n\pi}{\phi}}(qr)I_{\frac{n\pi}{\phi}}(qr_0) \right]
\end{eqnarray}

By Inverse Laplace transformation, $v$ can be obtained as
\begin{eqnarray}
    v &=& \frac{1}{2\phi Dt}\mathrm{e}^{-(r'^2 + r^2)/4Dt} \times\nonumber \\
      & & \Biggl(I_0\left(\frac{rr'}{2Dt}\right) + 
          2\sum^{\infty}_{n=1}\cos\frac{n\pi\theta}{\phi}\cos\frac{n\pi\theta_0}{\phi}
          I_{\frac{n\pi}{\phi}}\left(\frac{rr'}{2Dt}\right) \Biggr)
\end{eqnarray}
this is the Green's function of this system.


\end{document}
