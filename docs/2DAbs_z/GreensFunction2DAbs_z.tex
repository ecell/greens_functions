\documentclass{article}
\usepackage{indentfirst}
\usepackage{amsmath}
\usepackage{mathrsfs}

\begin{document}

\title{GreensFunction2DAbs}
\author{Toru Niina}
\maketitle

\section{Overview}

System contains only one particle. Initial position of the particle is $r=r_0$
(Position is descrived by the cylindorical coordinate,
$x = r\cos\theta, y = r\sin\theta$).

Boundary condition is Absorbance boundary $\psi(a,t) = 0$.
$\psi(r, t)$ means the probability of finding the particle in distance r and time t.

\subsection{Green's Function}

\begin{eqnarray}
    & & G(r, \theta, t | r', \theta', \tau) \nonumber\\
    &=& \frac{1}{\pi a^2}\sum^{\infty}_{m=1}
        \frac{J_0(r\alpha_{m0})J_0(r'\alpha_{m0})}{J^2_1(a\alpha_{m0})}
        \mathrm{e}^{-D\alpha^2_{m0} (t-\tau)} \nonumber \\
    &+& \frac{2}{\pi a^2}\sum^{\infty}_{n=1}\cos(n(\theta - \theta'))
        \sum^{\infty}_{m=1}
        \frac{4J_n(r\alpha_{mn})J_n(r'\alpha_{mn})}
             {(J_{n-1}(a\alpha_{mn})-J_{n+1}(a\alpha_{mn}))^2}
        \mathrm{e}^{-D\alpha^2_{mn} (t-\tau)} \nonumber
\end{eqnarray}
where $J_n$ is 1st kind Bessel's function and
$a\alpha_{mn}$ causes the 1st kind Bessel function to be zero like
$J_n(a\alpha_{mn}) = 0$. 

and the initial condition $F$ is

\begin{equation}
    F(r, \theta) = \delta(r - r_0)\delta(\theta)
\end{equation}
at $t=0$.

then the probability function is

\begin{eqnarray}
    P &=& \int^a_0 \int^{2\pi}_0 F(r', \theta')
                                 G(r, \theta, t | r', \theta', 0)
                                 r'dr'd\theta' \nonumber \\
      &=& \frac{1}{\pi a^2}\sum^{\infty}_{m=1}
                                 \frac{J_0(r\alpha_{m0})J_0(r_0\alpha_{m0})}
                                      {J^2_1(a\alpha_{m0})}
          \mathrm{e}^{-D\alpha^2_{m0} t} \nonumber \\
      &+& \frac{2}{\pi a^2}\sum^{\infty}_{n=1}\cos(n\theta)
          \sum^{\infty}_{m=1}
          \frac{4J_n(r\alpha_{mn})J_n(r_0\alpha_{mn})}
               {(J_{n-1}(a\alpha_{mn})-J_{n+1}(a\alpha_{mn}))^2}
          \mathrm{e}^{-D\alpha^2_{mn} t}
\end{eqnarray}

\subsection{derivation of the Green's function}

First, consider the point source in a 3 dimentional cylinder
and then integrate the function from $z = -\infty$ to $z = \infty$.

the diffusion equation is
\begin{eqnarray}
    D\nabla^2 P(r, \theta, t) &=& \frac{\partial P(r, \theta, t)}{\partial t} \nonumber \\
    P(a, \theta, t) &=& 0 \nonumber \\
    P(r, \theta, 0) &=& \delta(r')
\end{eqnarray}

In infinite medium, if there are probability source in $(x',y',z')$,

\begin{equation}
    u = \frac{1}{8\sqrt{\pi Dt}^{3}}\mathrm{e}^{(-R^2 + (z - z')^2) / 4Dt}
\end{equation}
where
\begin{equation}
    R = (x - x')^2 + (y - y')^2 = r'^2 + r^2 - 2rr'\cos(\theta - \theta')
\end{equation}

In below, let $z' = 0$.

using Laplace transformation, $\bar{u}$ can be written as
\begin{eqnarray}
    \bar{u}(p) &=& \frac{\mathrm{e}^{\sqrt{R^2 + z^2}}}
                      {4\pi D\sqrt{R^2 + z^2}} \nonumber \\
               &=& \frac{1}{2\pi^2 D}\int^{\infty}_{0}
                   \cos{\xi z}K_0(\eta R)d\xi \label{Subequ}
\end{eqnarray}
where $q = \sqrt{\frac{p}{D}}$, $\eta = \sqrt{(\xi^2 + q^2)}$.

And subsidery equation(with laplace transformation $t \to p$) is
\begin{equation}
    \frac{\partial^2 \bar{u}}{\partial r^2} +
    \frac{1}{r}\frac{\partial \bar{u}}{\partial r} +
    \frac{1}{r^2} \frac{\partial^2 \bar{u}}{\partial \theta^2} + 
    \frac{\partial^2 \bar{u}}{\partial z^2} -
    q^2 \bar{u} = 0
\end{equation}

Let the solution of this differencial equation $v = u + w$ 
and select $w$ to satisfy the boundary condision such that 
$v(a, \theta, t) = u(a, \theta, t) + w(a, \theta, t) = 0$.

Based on addition theorem for the Bessel function $K_0$,
\[
    K_0(\eta R)
    = \begin{cases}
        \sum_{n = -\infty}^\infty \cos n(\theta - \theta')I_n(\eta r)K_n(\eta r') & (r  < r') \\
        \sum_{n = -\infty}^\infty \cos n(\theta - \theta')I_n(\eta r')K_n(\eta r) & (r' < r )
      \end{cases}
\]
$\bar{u}$ where $r > r'$ can be written as
\begin{eqnarray}
    \bar{u} &=& \frac{1}{2\pi^2D}\int^{\infty}_{0}
                \cos{\xi z}K_0(\eta R)d\xi \nonumber \\
            &=& \frac{1}{2\pi^2D}\int^{\infty}_{0}
                \cos{\xi z}\sum_{n = -\infty}^\infty
                \cos n(\theta - \theta')I_n(\eta r')K_n(\eta r)d\xi\nonumber \\
            &=& \frac{1}{2\pi^2D}\sum_{n = -\infty}^\infty
                \cos n(\theta - \theta')\int^{\infty}_{0}
                \cos{\xi z}I_n(\eta r')K_n(\eta r)d\xi
\end{eqnarray}
and the value of $\bar{u}$ at $r = a$ is
\begin{equation}
    \bar{u}(a, \theta, t)
    = \frac{1}{2\pi^2D}\sum_{n = -\infty}^\infty
      \cos n(\theta - \theta')\int^{\infty}_{0}
      \cos{\xi z}I_n(\eta r')K_n(\eta a)d\xi
\end{equation}

$\bar{w}$ is also the solution of (\ref{Subequ}), so the form is same as $\bar{u}$.
In order $v$ to be zero in $r = a$, $\bar{w}$ become
\begin{equation}
    \bar{w} = \frac{-1}{2\pi^2D}\sum_{n = -\infty}^\infty
              \cos n(\theta - \theta')\int^{\infty}_{0}
              \cos{\xi z}\frac{I_n(\eta r)}{I_n(\eta a)}I_n(\eta r')K_n(\eta a)d\xi
\end{equation}

Simply add them, $\bar{v}$ can be obtained as
\begin{eqnarray}
    \bar{v} &=& \frac{1}{2\pi^2D}\sum_{n = -\infty}^\infty
                \cos n(\theta - \theta') \nonumber \\
            & & \int^{\infty}_{0}
                \cos{\xi z}\frac{I_n(\eta r)}{I_n(\eta a)}
                \left(I_n(\eta a)K_n(\eta r') - I_n(\eta r')K_n(\eta a)\right)d\xi
\end{eqnarray}

To calculate the integral
\begin{equation}
    \int^{\infty}_{0}
    \cos{\xi z}\frac{I_n(\eta r)}{I_n(\eta a)}
    \left(I_n(\eta a)K_n(\eta r') - I_n(\eta r')K_n(\eta a)\right)d\xi
\end{equation}
consider
\begin{equation}
    \int
    \mathrm{e}^{i\xi z}\frac{I_n(\eta r)}{I_n(\eta a)}
    \left(I_n(\eta a)K_n(\eta r') - I_n(\eta r')K_n(\eta a)\right)d\xi
\end{equation}
the contour is the real axis and a large semicircle in the upper plane($z > 0$).

The poles are the point that satisfies $I_n(\eta a) = 0$, 
at $\xi = i\sqrt{(q^2 + \alpha_{mn}^2)}$,
where $\alpha_{mn}$ satisfies $J_n(a\alpha_{mn}) = 0$.

Based on Residue theorem, the value of integral is
\begin{eqnarray}
    & & \int \mathrm{e}^{i\xi z}\frac{I_n(\eta r)}{I_n(\eta a)}
        \left(I_n(\eta a)K_n(\eta r') - I_n(\eta r')K_n(\eta a)\right)d\xi \nonumber \\
    &=& 2\pi i\sum Res(\mathrm{e}^{i\xi z}\frac{I_n(\eta r)}{I_n(\eta a)}
        \left(I_n(\eta a)K_n(\eta r') - I_n(\eta r')K_n(\eta a)\right))
\end{eqnarray}

The values of residues are, by definition,
\begin{eqnarray}
    & & Res(\mathrm{e}^{i\xi z}\frac{I_n(\eta r)}{I_n(\eta a)}
        \left(I_n(\eta a)K_n(\eta r') - I_n(\eta r')K_n(\eta a)\right)) \nonumber \\
    &=& \lim_{\xi \to \beta}
        (\xi - \beta)
        \mathrm{e}^{i\xi z}\frac{I_n(\eta r)}{I_n(\eta a)}
        \left(I_n(\eta a)K_n(\eta r') - I_n(\eta r')K_n(\eta a)\right)
\end{eqnarray}
here, $\beta = i\sqrt{(q^2 + \alpha_{mn}^2)}$.

Now, let $\xi = i\sqrt{s^2 + q^2}$ i.e., $\eta = is$, the residues are written as
\begin{equation}
    \lim_{s \to \alpha_{mn}}
    (s - \alpha_{mn})
    \mathrm{e}^{-\sqrt{s^2 + q^2} z}\frac{I_n(isr)}{I_n(isa)}
    \left(I_n(isa)K_n(isr') - I_n(isr')K_n(isa)\right)
    \frac{si}{\sqrt{s^2+q^2}} \label{residue_s}
\end{equation}
using the nature of Bessel function
\begin{eqnarray}
    I_n(z\mathrm{e}^{\pm\frac{\pi i}{2}}) &=& \mathrm{e}^{\pm\frac{n\pi i}{2}}J_n(z) \nonumber \\
    K_n(z\mathrm{e}^{\pm\frac{\pi i}{2}}) &=&
        \pm\frac{\pi i}{2}\mathrm{e}^{\mp\frac{n\pi i}{2}}(-J_n(z) \pm iY_n(z))
\end{eqnarray}
the value (\ref{residue_s}) can be written as
\begin{eqnarray}
    & & \lim_{s \to \alpha_{mn}}
        (s - \alpha_{mn})
        \mathrm{e}^{-\sqrt{s^2 + q^2} z}
        \frac{\mathrm{e}^{\frac{n\pi i}{2}}J_n(sr)}
        {\mathrm{e}^{\frac{n\pi i}{2}}J_n(sa)} \nonumber \\
    & & \Big(\mathrm{e}^{\frac{n\pi i}{2}}J_n(sa)
            \frac{\pi i}{2}\mathrm{e}^{-\frac{n\pi i}{2}}(-J_n(sr') + iY_n(sr')) \nonumber \\
    &-&     \mathrm{e}^{\frac{n\pi i}{2}}J_n(sr')
            \frac{\pi i}{2}\mathrm{e}^{-\frac{n\pi i}{2}}(-J_n(sa) + iY_n(sa))
        \Big) \frac{si}{\sqrt{s^2+q^2}}\nonumber \\
    &=& \lim_{s \to \alpha_{mn}}
        (s - \alpha_{mn})
        \mathrm{e}^{-\sqrt{s^2 + q^2} z}
        \frac{J_n(sr)}{J_n(sa)}\frac{\pi i}{2}\nonumber \\
    & & \Big(J_n(sa)(-J_n(sr') + iY_n(sr')) - J_n(sr')(-J_n(sa) + iY_n(sa))
        \Big) \frac{si}{\sqrt{s^2+q^2}}\nonumber \\
    &=& \lim_{s \to \alpha_{mn}}
        (s - \alpha_{mn})
        \frac{\mathrm{e}^{-\sqrt{s^2 + q^2} z}}{\sqrt{s^2+q^2}}
        \frac{J_n(sr)}{J_n(sa)}\frac{-s\pi}{2}\Big(iJ_n(sa)Y_n(sr') - iJ_n(sr')Y_n(sa)
        \Big)\nonumber \\
    &=& \lim_{s \to \alpha_{mn}}\frac{i\pi}{2}
        \frac{\mathrm{e}^{-\sqrt{s^2 + q^2} z}}{\sqrt{s^2+q^2}}
        \frac{(s - \alpha_{mn})}{J_n(sa)}sJ_n(sr)
        \Big(-J_n(sa)Y_n(sr') + J_n(sr')Y_n(sa)\Big)\nonumber \\
        \label{Lim_v}
\end{eqnarray}
The first part of (\ref{Lim_v}) is 
\begin{equation}
    \lim_{s \to \alpha_{mn}}\frac{i\pi}{2}
        \frac{\mathrm{e}^{-\sqrt{s^2 + q^2} z}}{\sqrt{s^2+q^2}}
  = \frac{i\pi}{2}
  \frac{\mathrm{e}^{-\sqrt{\alpha_{mn}^2 + q^2} z}}{\sqrt{\alpha_{mn}^2+q^2}}
\end{equation}
Then, second part of (\ref{Lim_v}) is, by definition of derivative, 
\begin{eqnarray}
    & & \lim_{s \to \alpha_{mn}} \frac{(s - \alpha_{mn})}{J_n(sa)} \nonumber \\
    &=& \lim_{s \to \alpha_{mn}}
        \frac{(as - a\alpha_{mn})}{a(J_n(sa) - J_n(a\alpha_{mn}))} \nonumber \\
    &=& \frac{1}{aJ'_n(a\alpha_{mn})}
\end{eqnarray}
The last part of (\ref{Lim_v}) is
\begin{eqnarray}
    & & \lim_{s \to \alpha_{mn}}sJ_n(sr)\Big(-J_n(sa)Y_n(sr') + J_n(sr')Y_n(sa)\Big) \nonumber \\
    &=& \lim_{s \to \alpha_{mn}}sJ_n(sr)J_n(sr')Y_n(sa) \nonumber \\
    &=& sJ_n(\alpha_{mn}r)J_n(\alpha_{mn}r')Y_n(\alpha_{mn}a) \label{last_limv}
\end{eqnarray}
and using
\begin{equation}
    zY_n(z) = \frac{-2}{\pi J'_n(z)} + \frac{zJ_n(z)Y'_n(z)}{J'_n(z)}
\end{equation}
(\ref{last_limv}) can be re-written as
\begin{eqnarray}
    & & \lim_{s \to \alpha_{mn}}sJ_n(sr)\Big(-J_n(sa)Y_n(sr') + J_n(sr')Y_n(sa)\Big) \nonumber \\
    &=& J_n(\alpha_{mn}r)J_n(\alpha_{mn}r')\frac{-2}{a\pi J'_n(a\alpha_{mn})}
\end{eqnarray}

Thus, the value of residue finally becomes
\begin{eqnarray}
    & & \lim_{s \to \alpha_{mn}}\frac{i\pi}{2}
        \frac{\mathrm{e}^{-\sqrt{s^2 + q^2} z}}{\sqrt{s^2+q^2}}
        \frac{(s - \alpha_{mn})}{J_n(sa)}sJ_n(sr)
        \Big(-J_n(sa)Y_n(sr') + J_n(sr')Y_n(sa)\Big) \nonumber \\
    &=& \frac{i\pi}{2}\frac{\mathrm{e}^{-\sqrt{\alpha_{mn}^2 + q^2} z}}
                           {\sqrt{\alpha_{mn}^2+q^2}}
        \frac{1}{aJ'_n(a\alpha_{mn})}
        J_n(\alpha_{mn}r)J_n(\alpha_{mn}r')
        \frac{-2}{a\pi J'_n(a\alpha_{mn})} \nonumber \\
    &=& \frac{-i}{a^2}\frac{\mathrm{e}^{-\sqrt{\alpha_{mn}^2 + q^2} z}}
                         {\sqrt{\alpha_{mn}^2+q^2}}
        \frac{J_n(\alpha_{mn}r)J_n(\alpha_{mn}r')}{\{J'_n(a\alpha_{mn})\}^2}
\end{eqnarray}

So $\bar{v}$ become
\begin{eqnarray}
\bar{v} &=& \frac{1}{2\pi^2D}\sum^\infty_{n = \infty}\cos n(\theta - \theta)
            2\pi i\sum_\alpha
            \frac{-i}{a^2}\frac{\mathrm{e}^{-\sqrt{\alpha_{mn}^2 + q^2} z}}
                     {\sqrt{\alpha_{mn}^2+q^2}}
            \frac{J_n(\alpha_{mn}r)J_n(\alpha_{mn}r')}{\{J'_n(a\alpha_{mn})\}^2} \nonumber \\
        &=& \frac{1}{\pi Da^2}\sum^\infty_{n = \infty}\cos n(\theta - \theta)
            \sum_\alpha
            \frac{\mathrm{e}^{-\sqrt{\alpha_{mn}^2 + q^2} z}}
                 {\sqrt{\alpha_{mn}^2+q^2}}
            \frac{J_n(\alpha_{mn}r)J_n(\alpha_{mn}r')}{\{J'_n(a\alpha_{mn})\}^2}
\end{eqnarray}

From inverse Laplace transformation,
\begin{equation}
    \mathscr{L}^{-1} \left\{\frac{\mathrm{e}^{-qx}}{q}\right\}(t)
  = \left(\frac{D}{\pi t}\right)^\frac{1}{2} \mathrm{e}^{-x^2/4Dt}
\end{equation}
thus $v$ is
\begin{equation}
    v = \frac{\mathrm{e}^{-z^2/4Dt}}{\pi a^2\sqrt{\pi Dt}}
        \sum^\infty_{n = -\infty}\cos n(\theta - \theta)
        \sum_\alpha
        \mathrm{e}^{-\alpha_{mn}Dt}
        \frac{J_n(\alpha_{mn}r)J_n(\alpha_{mn}r')}{\{J'_n(a\alpha_{mn})\}^2}
\end{equation}

Then integrate this from $z = -\infty$ to $z = + \infty$, using Gaussian integral, 
\begin{eqnarray}
    v &=& \frac{\sqrt{4Dt\pi}}{\pi a^2\sqrt{\pi Dt}}
          \sum^\infty_{n = -\infty}\cos n(\theta - \theta)
          \sum_\alpha
          \mathrm{e}^{-\alpha_{mn}Dt}
          \frac{J_n(\alpha_{mn}r)J_n(\alpha_{mn}r')}{\{J'_n(a\alpha_{mn})\}^2} \nonumber\\
      &=& \frac{2}{\pi a^2}
          \sum^\infty_{n = -\infty}\cos n(\theta - \theta)
          \sum_\alpha
          \mathrm{e}^{-\alpha_{mn}Dt}
          \frac{J_n(\alpha_{mn}r)J_n(\alpha_{mn}r')}{\{J'_n(a\alpha_{mn})\}^2}
\end{eqnarray}
this is the Green's function of this system.

\section{public member function}

\subsection{drawR(rnd, t) const}

drawR returns the distance between
the position of particle and the center of system in time t.
the return value $r$ satisfies
\begin{equation}
    \int^r_0 \int^{2\pi}_0 P(r', \theta', t) r'dr'd\theta' =
    \mathrm{p\_survival}(t) \times \mathrm{rnd}.
\end{equation}
p\_int\_r calculates the integration
$\int^r_0 \int^{2\pi}_0 P(r', \theta', t) r'dr'd\theta'$.

\subsection{drawTheta(rnd, r, t) const}

drawTheta returns the angle $\theta$ of the particle from initial position in time $t$ and radius $r$.

The return value $\theta$ satisfies the equation
\begin{equation}
    \cfrac{\int^\theta_0 P(r, \theta', t) d\theta'}
          {\int^{2\pi}_0 P(r, \theta', t) d\theta'} = \mathrm{rnd}
\end{equation}

this function should be called after drawR was called and
the value of $r$ determined.

p\_int\_theta calculate the integration
$\int^\theta_0 P(r, \theta', t) d\theta'$.
and p\_int\_2pi calculate the integration
$\int^\pi_0 P(r, \theta', t) d\theta'$.

\subsection{drawTime(rnd) const}

drawTime returns the time when the particle goes out of the boundary.
Return value $t$ satisfies $1 - \mathrm{p\_survival}(t) = \mathrm{rnd}$.

\subsection{p\_survival(t) const}

\begin{eqnarray}
    S(t) &=& \int^a_0 \int^{2\pi}_0 P(r, \theta, t) rdrd\theta \nonumber \\
         &=& \int^a_0 \frac{2}{a^2} \sum^{\infty}_{m=1}
             \frac{J_0(r\alpha_{m0}) J_0(r_0\alpha_{m0})}{J^2_1(a\alpha_{m0})}
             \mathrm{e}^{-\alpha^2_{m0}Dt} rdr \nonumber \\
         &+& \int^a_0 \int^{2\pi}_0 \frac{2}{\pi a^2}
             \sum^{\infty}_{n=1}\cos(n\theta)
             \sum^{\infty}_{m=1}
             \frac{4J_n(r\alpha_{mn})J_n(r_0\alpha_{mn})}
                  {(J_{n-1}(a\alpha_{mn})-J_{n+1}(a\alpha_{mn}))^2}
             \mathrm{e}^{(-D\alpha^2_{mn} t)} rdrd\theta \nonumber \\
         &=& \frac{2}{a} \sum^{\infty}_{m=1}
             \frac{J_0(r_0\alpha_{m0})}{\alpha_{m0} J_1(a\alpha_{m0})}
             \mathrm{e}^{-\alpha^2_{m0}Dt} \nonumber
\end{eqnarray}

\subsection{p\_int\_r(r, t) const}

\begin{eqnarray}
    & & \int^r_0 \int^{2\pi}_0 P(r', \theta, t) r'dr'd\theta \nonumber \\
    &=& \int^r_0 \frac{2}{a^2} \sum^{\infty}_{m=1}
        \frac{J_0(r'\alpha_{m0}) J_0(r_0\alpha_{m0})}{J^2_1(a\alpha_{m0})}
        \mathrm{e}^{-\alpha^2_{m0}Dt} r'dr' \nonumber \\
    &+& \int^r_0 \int^{2\pi}_0 
        \frac{2}{\pi a^2}\sum^{\infty}_{n=1}\cos(n\theta)
        \sum^{\infty}_{m=1}
        \frac{4J_n(r'\alpha_{mn})J_n(r_0\alpha_{mn})}
             {(J_{n-1}(a\alpha_{mn})-J_{n+1}(a\alpha_{mn}))^2}
        \mathrm{e}^{(-D\alpha^2_{mn} t)} r'dr'd\theta \nonumber \\
    &=& \frac{2}{a^2}\sum^\infty_{m=1} \frac{r}{\alpha_{m0}}
        \frac{J_1(r\alpha_{m0})J_0(r_0\alpha_{m0})}{J^2_1(a\alpha_{m0})}
        \mathrm{e}^{-\alpha^2_{m0}Dt}\nonumber
\end{eqnarray}

\subsection{p\_int\_theta(r, $\theta$, t) const}

Note: It is required that the value of $r$ is already determined by using drawR when this function is called.

This function returns the value of cummurative probability in $[0,\theta)$ for certain value of r.

\begin{eqnarray}
    & & \int^\theta_0 P(r, \theta', t) d\theta' \nonumber\\
    &=& \frac{\theta}{\pi a^2} \sum^{\infty}_{m=1}
        \frac{J_0(r\alpha_{m0}) J_0(r_0\alpha_{m0})}{J^2_1(a\alpha_{m0})}
        \mathrm{e}^{-\alpha^2_{m0}Dt} \nonumber \\
    &+& \int^{\theta}_0 \frac{2}{\pi a^2}\sum^{\infty}_{n=1}\cos(n\theta')\sum^{\infty}_{m=1}
        \frac{4J_n(r\alpha_{mn})J_n(r_0\alpha_{mn})}{(J_{n-1}(a\alpha_{mn})-J_{n+1}(a\alpha_{mn}))^2}
        \mathrm{e}^{-D\alpha^2_{mn} t} d\theta' \nonumber \\
    &=& \frac{\theta}{\pi a^2} \sum^{\infty}_{m=1}
        \frac{J_0(r\alpha_{m0}) J_0(r_0\alpha_{m0})}{J^2_1(a\alpha_{m0})}
        \mathrm{e}^{-\alpha^2_{m0}Dt} \nonumber \\
    &+& \frac{2}{\pi a^2} \sum^\infty_{n=1} \frac{\sin(n\theta)}{n}
        \sum^\infty_{m=1} \frac{4J_n(r\alpha_{mn})J_n(r_0\alpha_{mn})}{(J_{n-1}(a\alpha_{mn})-J_{n+1}(a\alpha_{mn}))^2}
        \mathrm{e}^{-D\alpha^2_{mn} t} \nonumber
\end{eqnarray}

\subsection{p\_int\_2pi(r, t) const}

Note: It is required that the value of $r$ is already determined by using drawR when this function is called.

This function returns the value of cummurative probability in $[0,2\pi)$ for certain value of r.

\begin{eqnarray}
    & & \int^{2\pi}_0 P(r, \theta, t) d\theta \nonumber\\
    &=& \frac{2}{a^2} \sum^{\infty}_{m=1}
        \frac{J_0(r\alpha_{m0}) J_0(r_0\alpha_{m0})}{J^2_1(a\alpha_{m0})}
        \mathrm{e}^{-\alpha^2_{m0}Dt} \nonumber \\
    &+& \int^{2\pi}_0 \frac{2}{\pi a^2}\sum^{\infty}_{n=1}\cos(n\theta)\sum^{\infty}_{m=1}
        \frac{4J_n(r\alpha_{mn})J_n(r_0\alpha_{mn})}{(J_{n-1}(a\alpha_{mn})-J_{n+1}(a\alpha_{mn}))^2}
        \mathrm{e}^{-D\alpha^2_{mn} t} d\theta \nonumber \\
    &=& \frac{2}{a^2} \sum^{\infty}_{m=1}
        \frac{J_0(r\alpha_{m0}) J_0(r_0\alpha_{m0})}{J^2_1(a\alpha_{m0})}
        \mathrm{e}^{-\alpha^2_{m0}Dt} \nonumber
\end{eqnarray}

\end{document}
